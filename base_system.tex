%!TEX root = ./Grimdark.tex
\chapter{Base System}
\section{Attributes}\label{Attributes}
Characters in Grimdark have 7 Attributes which define their relevant physical and psychical abilities.
Each Attribute is represented as a number from 0 to 100 - or potentially even higher.
The higher the Attribute the better.
A typical human will have scores in the thirties.
Without preternatural aid (like drugs, psychic powers, mutations or cybernetics) humans cannot have attributes greater than 60. 
All rolls in Grimdark make use of two associated Attributes - thus the attributes of your character are very important!
\begin{itemize}
	\item {\large\textbf{\underline{P}ower:}} Measures how powerful your character is - in body and mind.
	\begin{itemize}
		\item \textbf{Sample Uses:} lift heavy things, control automatic weapons, hit hard, intimidate others, break others will.
		\item \textbf{Archetypes:} Orks, Ogryn, Ganger, Interrogator, Commissar
	\end{itemize}

	\item {\large\textbf{\underline{F}inesse:}} Measures how well your character does delicate work.
	\begin{itemize}
		\item \textbf{Sample Uses:} make a hard target, one shot one kill, move silent, craft or manipulate delicate items, manipulate others
		\item \textbf{Archetypes:} Eldar, Assassin, Sniper, Mechanicus Magos, Diplomat
	\end{itemize}

	\item {\large\textbf{\underline{T}oughness:}} Measures how much of a beating you can take.
	\begin{itemize}
		\item \textbf{Sample Uses:} endure pain, keep doing strenuous work, resist poison and illness, recover fast
		\item \textbf{Archetypes:} Orks, Ogryn, Ganger, Soldiers
	\end{itemize}

	\item {\large\textbf{\underline{I}nstinct:}} Measures how keen your senses are and how well you react to any given situation
	\begin{itemize}
		\item \textbf{Sample Uses:} read the mood of others, detect danger (traps, ambushes, followers, attacks), react fast
		\item \textbf{Archetypes:} Eldar, Orks, Ogryn, Space Marines, Inquisitor, Scout
	\end{itemize}

	\item {\large\textbf{\underline{S}marts:}} Measures how much the character knows.
	\begin{itemize}
		\item \textbf{Sample Uses:} know potential useful tidbits, know what (and what not) to say, do research, treat wounds, handle tech
		\item \textbf{Archetypes:} Eldar, Adept, Mechanicus Magos
	\end{itemize}

	\item {\large\textbf{\underline{C}harisma:}} Measures how well the character can express himself and work with people.
	\begin{itemize}
		\item \textbf{Sample Uses:} preach, command, talk others into something, discussions, trading, information gathering, flirting
		\item \textbf{Archetypes:} Eldar, Officer, Quartermaster, Diplomat, Priest, Rogue Trader
	\end{itemize}

	\item {\large\textbf{\underline{W}ill:}} Measures the strength of the characters will.
	\begin{itemize}
		\item \textbf{Sample Uses:} keep cool under pressure, resist mental influence, keep going when exhausted, psyker powers
		\item \textbf{Archetypes:} Eldar, Inquisitor, Space Marine, Commissar, Psyker, Witch
	\end{itemize}
\end{itemize}


\subsection{Attribute Modifier}
Some situations allow a character to add a modifier based on his Attribute to an effect, e.g. when hitting an enemy with a melee weapon you may add your Power Modificator (P-Mod) on top of the damage dealt!
Usually, the Attribute Modifier (or simply Mod) is the first digit of the respective attribute - or expressed otherwise, the Mod is the Attribute divided by 10. 

Effect that make use of such a Mod, refer to it as X-Mod, where the X is the first letter of the used Attribute.
So P-Mod refers to the Power Modifier and W-Mod to your Will Modifier!

\textit{Example: If your character's Strength score is 37, then his S-Mod will be 3.}

\section{Aptitudes}\label{Aptitudes}
Where Attributes define the \textit{current} abilities of your character, Aptitudes represent your character's \textit{potential}.
Aptitudes define how easy your character can learn new skills, professions and talents associated with the Aptitudes.
While every character in Grimdark can learn any skill or talent, a character with the respective Aptitudes will have a much easier time doing so.
In game terms, the number of matching Aptitudes determines the XP cost to learn something new.

When learning a new skill or talent, check its referenced Aptitudes. 
If your character has both Aptitudes he will have a very easy time learning it.
Do you only have one of them, you will do okay.
But if you have neither, your character will find it pretty hard (but nor impossible!) to learn this new ability.

Grimdark uses a total of 16 Aptitudes, grouped by Attribute and Non-Attribute Aptitudes.
The former group consists of Aptitudes which are named after one of the 7 Attributes and represent those Attributes that a character feels a natural connection with: Power might come naturally to an Athlete while a good connection with Charisma makes for a good Spokesman.
The latter group does represent more specific yet broad concepts a character might feel particularly attuned to:
\begin{itemize}
	\item \textbf{Offense:} Offensive behavior is second nature for you. Why knock at a door if you can kick it in? Why aim a single round when a full clip is ready? Why debate with someone you can cow into submission?
	\item \textbf{Defense:} Defensive behavior is of great concern to the character. Making yourself a hard target during a gunfight or parry the blow of an attacker is a good first step to survive after all. Resisting manipulations and temptation is just as well.
	\item \textbf{Subtlety:} You prefer the delicate approach over brute force. A single well placed round might take down an opponent which could withstand a full salvo of ill-aimed shots. Graceful maneuvers are their own reward - you don't want to be mistaken for an Ork after all, right?
	\item \textbf{Fieldcraft:} You are a crafty person. Whether this is expressed by your great skill with an scalpel in a field surgery, a multi-tool in the repair bay, with tripwire and explosives, in the taming den of a wild beast or by setting up a primitive shelter in the woods: you are up to the task!
	\item \textbf{Knowledge:} You know stuff. And you love to learn new stuff. Potentially dangerous and heretical as it may be, it may also prove very helpful.
	\item \textbf{Leadership:} You are a born leader. Whether by powerful speech, religious fervor, exemplary behavior, trickery or intimidation: you can lead your fellows and make them act at your command.
	\item \textbf{Social:} The thoughts, fears and desires of your fellows are an open book for you. Whether you seek to understand them, see through theirs lies, manipulate them or make them belief your every word is up to you...
	\item \textbf{Tech:} You have a deep connection to technology and all of its mysteries. You can easily handle machinery, understand and solve operational problems and may even attempt the great task of creating new devices from scratch or understand archeotech!
	\item \textbf{Psyker:} You have an (un-)fortunate connection to the warp, with all the powers and dangers that come with it. You may be a full fledged psyker and able to call on its powers or you may only be slightly more susceptible for its influence and thus able to feel nearby warp phenomena or entities.
\end{itemize}




\section{The Core System} % (fold)
\label{sec:base_system}
Grimdark uses a universal roll system. Every check in the game uses the same base system. The following chapter will describe it in detail.
\subsection{The Base Roll}
Whenever a character uses a skill (in a non-trivial way), the player will roll a skill check.
The check uses two d100 dice rolls, one for each attribute linked to the skill in question.
If the roll result is equal to or lower than the target number (attribute value plus potential modifiers), it is considered a pass.

\subsection{Trinary Outcome}
A Base Roll can have three possible outcomes: \\
\begin{itemize}
	\item \textbf{0 Pass - Failure:} Both rolls are above their target numbers and the check fails and causes a complication.
	\item \textbf{1 Pass - Partial Success (PS):} One of the rolls is above the target number, the other a pass. The check achieved a Partial Success, that is the goal is reached but at a cost. The cost is decided by the GM according to the situation but the group is encouraged to provide input if there is no obvious cost. See the example below for two possible costs.
	\item \textbf{2 Pass - Full Success (FS):} Both rolls are passes. The check achieved a Full Success. The goal is achieved without problem.
\end{itemize}

\begin{figure}[ht]
	\begin{DndSidebar}{Example Skill Check}
	\textit{Sister Aryanna of the Ordo Dialogis attempts to calm down Inquisitors Zu-Lin Yu and Otto Brecht who have come close to open hostility over a debate of how to handle the survivors of an Imperial Guard regiment fighting off a daemonic incursion...}
	\begin{itemize}
		\item \textbf{Skill:} Pretty Words
		\item \textbf{Skill Modifier:} +5
		\item \textbf{Attributes:} Fellowship 47, Intelligence 34
		\item \textbf{Target Numbers:} 52 (47+5), 39 (34+5)
		\item \textbf{Rolls:} 64 (no pass), 23 (pass)
		\item \textbf{Result:} Partial Success
	\end{itemize}
	The player rolls two d100. The first comes up with a 64 and the second with a 23.
	This means that the check resulted in a Partial Success, since only one roll was below the target number (the second one).
	The goal will be achieved (stopping the Inquisitors from starting a internecine war) - but at a cost. For most social skills, a Partial Success will leave the a bad aftertaste with the target of the skill, making it potentially more difficult to interact with in the future. Another common complication of social skills would be the target challenging the character to a debate over the matter, resulting in a second opposed test. \\
	\textit{Sister Aryanna speaks up and argues that executing the soldiers, who are by now very well trained to detect and fight the neverborn, would be massive waste of Imperial assets and simply cannot be tolerated at the current stage of the war. While her temper rose quickly and flamed up brightly over yet another debate between Yu and Brecht, her arguments are sound and not easily squashed even by a hardline puritan like Brecht.\\
	Inquisitor Brecht will surely remember her interdiction - but the argument has been settled without bloodshed.}
	\end{DndSidebar}
\end{figure}

\subsection{Fate, Luck \& Doom}
Sometimes characters just get lucky, a blindly fired shot into a smoke screen hitting an enemy in the head or unknowingly addressing the weak spot during a discussion. And at other times, the entire universe just seem to have met in a murky back room of a shady bar to plot against them.

Grimdark uses the term Fate for this. Fate comes into play when a character rolls a skill (or profession) check and one (or both) dice come up with double-digit results (11, 22, ..., 99).
Whether Fate is good or bad, is determined by the overall outcome of the roll - think of it as an catalyst. 
Good things get better. 
Bad things get worse.
Getting Fate on a Full Success will result in something truly spectacular, while getting Fate on a Fail mean you are in big trouble this time for sure...

Some skills or talents will provide special rules or guidelines for how to treat Fate when it comes up as part of its use.
Generally, GM and players should work together to make fateful rolls remarkable and memorial - especially in the case of double Fate (both rolls are double-digits) as those are very, very rare indeed.


\subsection{The Opposed Roll}
\label{opposed_roll}
Many situations in Grimdark are resolved by making an opposed roll between two parties.
This is especially true for inherently competitive situations like combat, social manipulation or contests.\\
An opposed roll works as follows:
\begin{enumerate}
	\item The active party chooses the skill first.
	\item The passive party chooses a fitting skill in response.
	\item Both sides roll their respective checks as normal and get their trinary outcome.
	\item The outcomes of both parties is compared to get the result of the opposed test.
	\item Apply Fate of the winning party as normal.
\end{enumerate}

The results of an opposed roll is determined by the following table:
\begin{tabular}{l | l | l}
	\textbf{Active} &  \textbf{Passive} & \textbf{Outcome} \\ 
	FS & Fail & CS\\
	FS & PS & FS \\
	FS & FS & Draw \\
	PS & Fail & PS \\
	PS & PS & Draw \\
	PS & FS & Fail \\
	Fail & Fail & Draw\\
	Fail & PS & Fail \\
	Fail & FS & CF\\
\end{tabular}

While this might appear complicated at first, there are three easy to remember rules that determine the outcome:
\begin{enumerate}
	\item When the active party gets more passes than the passive party, the outcome of the active party is the end result.
	\item When both parties get the same outcome, there is a draw. If one party has Fate on her roll, the result of that party determines the end result. If Fate does not play a role (or both parties get an equal amount of Fate dice), the outcomes of a draw is determined by the used skill of the active party (see section \ref{skill_types}).
	\item If one party gets a FS and the other a Fail, a critical effect happens. 
	\begin{itemize}
		\item Critical Success (CS): Happens when the active party gets a FS and the passive party a Fail. Counts as a FS but with enhanced effects determined by the GM.
		\item Critical Failure (CF): Happens when the active party gets a Fail and the passive party a FS. Counts as a Fail but with additional or worsened complication(s) determined by the GM.
	\end{itemize}

\end{enumerate}



% section base_system (end)
