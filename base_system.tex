%!TEX root = ./Grimdark.tex
\chapter{Base System}
\section{Attributes}\label{Attributes}
	Characters in Grimdark have 7 Attributes which define their relevant physical and psychical abilities.
	Each Attribute is represented as a number from 0 to 100 - or potentially even higher.
	The higher the Attribute the better.
	A typical human will have scores in the thirties.
	Without preternatural aid (like drugs, psychic powers, mutations or cybernetics) humans cannot have attributes greater than 60. 
	All rolls in Grimdark make use of two associated Attributes - thus the attributes of your character are very important!
	\begin{itemize}
		\item {\large\textbf{\underline{B}rutality:}} Measures how brutal your character can be - in body and mind.
		\begin{itemize}
			\item \textbf{Sample Uses:} lift heavy things, control automatic weapons, hit hard, intimidate others, break others will (and doors).
			\item \textbf{Archetypes:} Orks, Ogryn, Ganger, Interrogator, Commissar
		\end{itemize}

		\item {\large\textbf{\underline{P}recision:}} Measures how well your character does delicate work.
		\begin{itemize}
			\item \textbf{Sample Uses:} one shot one kill, move silent, craft or manipulate delicate items, manipulate others, balance
			\item \textbf{Archetypes:} Eldar, Assassin, Sniper, Mechanicus Magos, Diplomat
		\end{itemize}

		\item {\large\textbf{\underline{S}peed:}} Measures how fast your character acts (and thinks).
		\begin{itemize}
			\item \textbf{Sample Uses:} run, make yourself a hard target, reflexive attacks, operate fast moving vehicles
			\item \textbf{Archetypes:} Eldar, Assassin, Pilot, Lexmechanicus
		\end{itemize}

		\item {\large\textbf{\underline{T}oughness:}} Measures how much of a beating you can take.
		\begin{itemize}
			\item \textbf{Sample Uses:} endure pain, keep doing strenuous work, resist poison and illness, recover fast
			\item \textbf{Archetypes:} Orks, Ogryn, Ganger, Soldiers
			\item \textbf{Special:} influences your Resilience, that is the amount of damage you can endure before suffering a Snap (see \ref{Resilience})
		\end{itemize}

		\item {\large\textbf{\underline{I}nstinct:}} Measures how keen your senses are and how well you react to any given situation
		\begin{itemize}
			\item \textbf{Sample Uses:} read the mood of others, detect danger (traps, ambushes, followers, attacks), gut feelings, navigate unknown terrain
			\item \textbf{Archetypes:} Eldar, Orks, Ogryn, Space Marines, Inquisitor, Scouts
		\end{itemize}

		\item {\large\textbf{\underline{C}unning:}} Measures how clever and resourceful your character is.
		\begin{itemize}
			\item \textbf{Sample Uses:} know what (and what not) to say, do research, treat wounds, handle tech, know potential useful tidbits
			\item \textbf{Archetypes:} Eldar, Adept, Mechanicus Magos, (some) Orks
		\end{itemize}

		\item {\large\textbf{\underline{W}ill:}} Measures the strength of the character's will.
		\begin{itemize}
			\item \textbf{Sample Uses:} keep cool under pressure, resist mental influence, keep going when exhausted, psyker powers
			\item \textbf{Archetypes:} Eldar, Inquisitor, Space Marine, Commissar, Psyker, Witch
			\item \textbf{Special:} influences your Resilience together with Toughness
		\end{itemize}
	\end{itemize}


	\subsection{Attribute Modifier}
		Some situations allow a character to add a modifier based on his Attribute to an effect, e.g. when hitting an enemy with a melee weapon you may add your Brutality modificator (B-Mod) on top of the damage dealt!
		Usually, the Attribute Modifier (or simply Mod) is the first digit of the respective attribute - or expressed otherwise, the Mod is the Attribute divided by 10. 

		Effects that make use of such a Mod, refer to it as X-Mod, where the X is the first letter of the used Attribute.
		So B-Mod refers to the Brutality Modifier and W-Mod to your Will Modifier!

		\textit{Example: If your character's Will score is 37, then his W-Mod will be 3.}

\section{Aptitudes}\label{Aptitudes}
	Where Attributes define the \textit{current} abilities of your character, Aptitudes represent your character's \textit{potential}.
	Aptitudes define how easy your character can learn new skills, professions and talents associated with the Aptitudes.
	While every character in Grimdark can learn any skill or talent, a character with the respective Aptitudes will have a much easier time doing so.
	In game terms, the number of matching Aptitudes determines the XP cost to learn something new.

	There are two types of Aptitudes: Attribute and Group Aptitudes.
	\begin{itemize}
		\item Attribute Aptitudes share the name of a Attribute (e.g. Brutality or Cunning). For each Attribute there is a matching Aptitude. Having an Attribute Aptitude means that your character will find all skills relying on this Attribute easy to master. Some people simple find it easy to learn new things from books (which would be represented by the Smarts Aptitude) while others excel at quickly training a new combat technique (which would be represented by either a Power or Finesse Aptitude).
		\item Group Attributes share the name of a Profession Group (see section \ref{profession_overview_grouped}), that is: Physical, Intellectual, Social and Instinctual. A character with a certain Group Attribute, will find all Professions (and thus skills) within this group easy to learn and master.
	\end{itemize}

	When learning a new skill or talent, check the Attributes it relies upon and he Profession group it is part of, then compare these to your Aptitudes.
	If your character has two (or more) matching Aptitudes he will have a very easy time learning it.
	Do you only have one of them, you will do okay.
	But if you have neither, your character will find it pretty hard (but nor impossible!) to learn this new ability.


\section{The Core System} % (fold)
\label{sec:base_system}
Grimdark uses a universal roll system. Every check in the game uses the same base system. The following chapter will describe it in detail.
\subsection{The Base Roll}
Whenever a character uses a skill (in a non-trivial way), the player will roll a skill check.
The check uses two d100 dice rolls, one for each attribute linked to the skill in question.
If the roll result is equal to or lower than the target number (attribute value plus potential modifiers), it is considered a pass.

\subsection{Trinary Outcome}
A Base Roll can have three possible outcomes: \\
\begin{itemize}
	\item \textbf{0 Pass - Failure:} Both rolls are above their target numbers and the check fails and causes a complication.
	\item \textbf{1 Pass - Partial Success (PS):} One of the rolls is above the target number, the other a pass. The check achieved a Partial Success, that is the goal is reached but at a cost. The cost is decided by the GM according to the situation but the group is encouraged to provide input if there is no obvious cost. See the example below for two possible costs.
	\item \textbf{2 Pass - Full Success (FS):} Both rolls are passes. The check achieved a Full Success. The goal is achieved without problem.
\end{itemize}

\begin{figure}[ht]
	\begin{DndSidebar}{Example Skill Check}
	\textit{Sister Aryanna of the Ordo Dialogis attempts to calm down Inquisitors Zu-Lin Yu and Otto Brecht who have come close to open hostility over a debate of how to handle the survivors of an Imperial Guard regiment fighting off a daemonic incursion...}
	\begin{itemize}
		\item \textbf{Skill:} Pretty Words
		\item \textbf{Skill Modifier:} +5
		\item \textbf{Attributes:} Fellowship 47, Intelligence 34
		\item \textbf{Target Numbers:} 52 (47+5), 39 (34+5)
		\item \textbf{Rolls:} 64 (no pass), 23 (pass)
		\item \textbf{Result:} Partial Success
	\end{itemize}
	The player rolls two d100. The first comes up with a 64 and the second with a 23.
	This means that the check resulted in a Partial Success, since only one roll was below the target number (the second one).
	The goal will be achieved (stopping the Inquisitors from starting a internecine war) - but at a cost. For most social skills, a Partial Success will leave the a bad aftertaste with the target of the skill, making it potentially more difficult to interact with in the future. Another common complication of social skills would be the target challenging the character to a debate over the matter, resulting in a second opposed test. \\\noindent
	\textit{Sister Aryanna speaks up and argues that executing the soldiers, who are by now very well trained to detect and fight the neverborn, would be massive waste of Imperial assets and simply cannot be tolerated at the current stage of the war. While her temper rose quickly and flamed up brightly over yet another debate between Yu and Brecht, her arguments are sound and not easily squashed even by a hardline puritan like Brecht.\\\noindent
	Inquisitor Brecht will surely remember her interdiction - but the argument has been settled without bloodshed.}
	\end{DndSidebar}
\end{figure}

\subsection{Fate, Luck \& Doom}
Sometimes characters just get lucky, a blindly fired shot into a smoke screen hitting an enemy in the head or unknowingly addressing the weak spot during a discussion. And at other times, the entire universe just seem to have met in a murky back room of a shady bar to plot against them.

Grimdark uses the term Fate for this. Fate comes into play when a character rolls a skill (or profession) check and one (or both) dice come up with double-digit results (11, 22, ..., 99).
Whether Fate is good or bad, is determined by the overall outcome of the roll - think of it as an catalyst. 
Good things get better. 
Bad things get worse.
Getting Fate on a Full Success will result in something truly spectacular, while getting Fate on a Fail mean you are in big trouble this time for sure...

Some skills or talents will provide special rules or guidelines for how to treat Fate when it comes up as part of its use.
Generally, GM and players should work together to make fateful rolls remarkable and memorial - especially in the case of double Fate (both rolls are double-digits) as those are very, very rare indeed.


\subsection{The Opposed Roll}
\label{opposed_roll}
Many situations in Grimdark are resolved by making an opposed roll between two parties.
This is especially true for inherently competitive situations like combat, social manipulation or contests.\\
An opposed roll works as follows:
\begin{enumerate}
	\item The active party chooses the skill first.
	\item The passive party chooses a fitting skill in response.
	\item Both sides roll their respective checks as normal and get their trinary outcome.
	\item The outcomes of both parties is compared to get the result of the opposed test.
	\item Apply Fate of the winning party as normal.
\end{enumerate}

The results of an opposed roll is determined by the following table:
\begin{DndTable}[header=Opposed Role Outcomes]{XXX}
    \textbf{Active}  & \textbf{Passive} & \textbf{Outcome}\\
	FS & Fail & CS\\
	FS & PS & FS \\
	FS & FS & Draw \\
	PS & Fail & PS \\
	PS & PS & Draw \\
	PS & FS & Fail \\
	Fail & Fail & Draw\\
	Fail & PS & Fail \\
	Fail & FS & CF\\
\end{DndTable}

While this might appear complicated at first, there are three easy to remember rules that determine the outcome:
\begin{enumerate}
	\item When the active party gets more passes than the passive party, the outcome of the active party is the end result.
	\item When both parties get the same outcome, there is a draw. If one party has Fate on her roll, the result of that party determines the end result. If Fate does not play a role (or both parties get an equal amount of Fate dice), the outcomes of a draw is determined by the used skill of the active party (see section \ref{skill_types}).
	\item If one party gets a FS and the other a Fail, a critical effect happens. 
	\begin{itemize}
		\item Critical Success (CS): Happens when the active party gets a FS and the passive party a Fail. Counts as a FS but with enhanced effects determined by the GM.
		\item Critical Failure (CF): Happens when the active party gets a Fail and the passive party a FS. Counts as a Fail but with additional or worsened complication(s) determined by the GM.
	\end{itemize}

\end{enumerate}



% section base_system (end)
