%!TEX root = ./Grimdark.tex
\chapter{Professions \& Skills}
Grimdark's skill system is divided into so-called Professions, which group associated skills together. Raising a certain skill will also raise the profession bonus, which can be used in place of any profession skill. 
This represents the fact that specialists are also (somewhat) proficient in the general field.\\
A renown surgeon for cybernetic replacements is certainly also quite capable at first aid treatments after all, and a famous climber is quite likely a dangerous wrestler as well...
\begin{figure}[ht]
	\begin{DndReadAloud}
	\textbf{Generalist or Specialist Path?}

	The number of skills featured in Grimdark is quite large (and might even grow in your game!), so it is easy to feel overhwelmed.
	It is totally acceptable to completely set aside skills for your first (or any) character and only use the more general Professions instead! Grimdark calls this the \textbf{Generalist} path.
	Whenever the GameMaster calls for a skill check, he should also announce the respective Profession.
	If you choose to follow the Generalist path, you will just make the respective Profession check and you're good.

	Should you prefer the delve deeper into Grimdark's skill system and manage each individual skill, you are set to follow the \textbf{Specialist} path.
	The Specialist path provides the option to fiddle with every last bit of your character, and can be seen as a tool for advanced players (or those who like a complex system) to tweak and customize their characters even further and make them truly unique.
	\end{DndReadAloud}
\end{figure}

\section{Professions} % (fold)
\label{sec:professions}
A Profession represent the generalist ability of a character in a certain broad area of expertise.
Each Profession covers a range of skills which are somewhat related.
In addition, they provide a fast and loose overview of the character.
Is he a brutal underhive ganger with high scores in the Betrayer, Athlete and Warrior professions?
Or an ancient tech priest with a more accumulated knowledge in his left (augmented) toe, than a handful of highborn nobles put together, reflected by his  Tinkerer, Technologist and Logic professions?
\begin{figure}[ht]
	\begin{DndSidebar}{Profession Example}
	\textit{Sister Aryanna of the Ordo Dialogis and her death-cult comrade Krill'Ta are attempting to infiltrate a gathering of noblewomen of the Hive World Olrabus, suspecting the machinations of a genestealer cult in its midst. As a distrustful guard is denying them entrance, a hasty debate between the two members of the Inquisitors retinue using Ordo hand signs erupts. Should Aryanna attempt to tell a convoluted (and utterly crafted) story to convince the guard - or should Krill'Ta attempt to seduce the guard with her death world charm?\\
	Both would be valid approaches, but seducing the guard may take Krill'Ta out of the next scene and leave the sister alone in the midst of the noblewomen. On the other hand, sister Aryanna isn't particular known for her abilities to lie - an ability which is far from well-perceived by her Ordo after all...\\}
	\end{DndSidebar}
\end{figure}

One central decision of Grimdark, contrary to many other game systems, is the fact that it does not assume that player characters or even a group of player characters, cover all or even most professions. 
There are many different ways to tackle any given situation scattered over the different professions that may appear redundant on a purely mechanical take.
But each profession has its very own take and spin on "their" solution to a problem, along with certain role-play implications, limitations and situational modifiers.
Thus a certain given situation may be much easier to solve by one profession than by another - yet they all have their uses, without forcing someone to break out of character to tackle a certain solution.

On a mechanical level, Professions work (mostly) like skills.
You can use them instead of any related skill.
The main difference between skills and Professions, is the fact that you cannot raise Professions directly.
The roll modifier is instead determined by the current Profession rating (see below).

\begin{figure}[hb]
	\begin{DndReadAloud}
	\textbf{Design Decision: Professions}
	\begin{itemize}
		\item \textbf{Facilitate Player Creativity}: If a player has an idea, which is not covered by the existing skills, it is usually easy to discern the Profession to which the described action relates. Having the option to roll a Profession check in such a case, allows for great player creativity. GM and Players are encouraged to develop new skills for commonly used actions of players that are not foreseen by Grimdark.
		\item \textbf{Make Everyone a Generalist}: In many skill-based game systems, players may end up with specialist characters which are very proficient in skill A but absolutely untrained in skill B, although the two skills are actually quite closely related. This does not make a whole lot of sense, as every specialist is also necessarily a generalist in his field of purview - at least to some degree.
		\end{itemize}
	\end{DndReadAloud}
\end{figure}

\subsection{The Profession Rating}
Each profession has a certain Profession rating, representing the total ability of the character in that Profession. 
This rating \textbf{increases by 2 for each new skill level learned} in an associated skill of the profession.

\subsection{The Profession Check}
A Profession check works just like a normal skill check with a single difference.
Instead of using the respective skill modifier, it uses the Profession Rating.
Everything else works the same, including the used attributes and situational modifiers of the respective skill.

\begin{figure}[hb]
	\begin{DndSidebar}{Profession Check Example}
	\textit{Inquisitor Otto Brecht has been trapped into a hive corridor with all doors locked. The straight forward Inquisitors decides to try and force open one of the doors with his own strength, hoping to still remain his covert approach that way. If that fails, he can still blast open the doors with a grenade after all ...\\
	Brecht is not particularly skilled at breaking things- he relies on the brute Nic'Or for such things - but he is keeping himself in top physical condition and is tough as nail. He should have a good enough chance to break open the doors...}\\
	That is, Brecht does not have the Break skill of the Athlete Profession, but he does have two ranks each in the Triathlete and Endure skills of that Profession. Thus, his Athlete Profession rating is: $2*4=8$.
		\begin{itemize}
		\item \textbf{Skill / Profession:} Break / Athlete
		\item \textbf{Skill Modifier:} 0 (untrained => use Profession)
		\item \textbf{Profession Rating:} 8
		\item \textbf{Attributes:} Strength 37, Toughness 44
		\item \textbf{Modifiers:} Locked Door (-10)
		\item \textbf{Target Numbers:} 35 (37+8-10), 42 (44+8-10)
		\item \textbf{Rolls:} 77 (no pass), 43 (no pass)
		\item \textbf{Result:} Failure
	\end{itemize}
	\textit{But the door won't budge. Going loud it is then. So be it. He never liked sneaking around anyway...}
	\end{DndSidebar}
\end{figure}

\subsection{The Professions}
Grimdark uses the following professions:
\begin{itemize}
	\item \textbf{\nameref{Acrobat}:} covers skills that require a high level of body control, tension and swiftness.
	\item \textbf{\nameref{Athlete}:} covers skills that require a high level of physical strength and endurance.
	\item \textbf{\nameref{Believer}:} covers skills that require a deep conviction in a belief.
	\item \textbf{\nameref{Betrayer}:} covers skills that deceive others or break common law.
	\item \textbf{Commander:} covers skills that lead others by example, pressure or wit.
	\item \textbf{Enforcer:} covers skills that cow others into submission or fear.
	\item \textbf{Ghost:} covers skills that allow the character to stay hidden.
	\item \textbf{Gunner:} covers skills of ranged weapon fighting.
	\item \textbf{Medic:} covers skills needed to keep others alive.
	\item \textbf{Navigator:} covers skills that prevent you from getting lost.
	\item \textbf{Observer:} covers skills that require a high level of alertness and perceptiveness.
	\item \textbf{Operator:} covers skills that allow the operation of vehicles and its functions.
	\item \textbf{Psyker:} covers the arcane skills only open to the witch.
	\item \textbf{Rationalist:} covers skills that require a high level of intelligence and clear thought.
	\item \textbf{Spokesman:} covers skills that rely on communicating with others.
	\item \textbf{Survivalist:} covers skills that allow characters to survive in any environment on their own.
	\item \textbf{Technologist:} covers skills that allow the proper use, understanding and even crafting of advanced tech.
	\item \textbf{Tinkerer:} covers skills that allow the proper use of common and low tech.
	\item \textbf{Trader:} covers skills related to money, property and value.
	\item \textbf{Warrior:} covers skills of close-quarter fighting.
\end{itemize}

\subsubsection{Acrobat}\label{Acrobat}
\textit{The character shows a high level of body control and tension. 
His movements are fluid, every step measured, every gesture a study in precision.
He is likely to possess quite some strength without appearing swollen with muscle and may appear lean or even slim without being sunken or weak.
The character is likely to be considered handsome or even beautiful depending on the local beauty ideals and other people might react accordingly.}
\begin{itemize}
	\item \textbf{Aptitudes:} Agility, Finesse
	\item \textbf{Common Attributes:} Agility, Strength, Toughness, Willpower
	\item \textbf{Skills:} Balancing, Contortionist, Dancing, Hard Target, Leap, Maneuvering
	\item \textbf{Sample Consequences:} 
	\begin{itemize}
		\item Twisted Ankle: Cannot run or charge for the remainder of the scene.
		\item Snapped Tendon: -5 on any agility based check until healed.
		\item Bruise: 2d5 points of Resilience damage.
		\item Trip: Fall prone.
		\item Wide Open: Enemies gain +10 to attack the character until his next turn.
	\end{itemize}
	\item \textbf{Common Modifiers:}
	\begin{itemize}
		\item High Gravity: -15
		\item Zero Gravity: -10
		\item No Armor worn: +15
		\item Bulky Weapon: -5
		\item Slippery Ground (Water, Ice, Oil): -5 (or worse)
	\end{itemize}
\end{itemize}

\subsubsection{Athlete}\label{Athlete}
\textit{The character is visibly strong and tough.
He likely has great muscle mass and little to no fat deposits.
Most people will expect the character to be a good warrior - which may influence the way they perceive the character and react to him.
He may be physically intimidating to less strong characters even without meaning to.}
\begin{itemize}
	\item \textbf{Aptitudes:} Strength OR Toughness, Offense
	\item \textbf{Common Attributes:} Strength, Toughness, Willpower, Weapon Skill, Agility
	\item \textbf{Skills:} Block, Brawl, Break, Endure, Heave, Throw, Triathlete
	\item \textbf{Sample Consequences:} 
	\begin{itemize}
		\item Torn Muscle: Cannot perform any Athlete skill checks until healed.
		\item Bruise: 2d5 Resilience damage
		\item Fractured Bone: Increases resilience damage taken by 2 until healed. Cumulative.
		\item Push-Back: The character is pushed back by 1d5 meters.
		\item Contusion: -5 on any strength based check.
	\end{itemize}
	\item \textbf{Common Modifiers:}
	\begin{itemize}
		\item Solid Ground: +5
		\item Unsteady Ground (Pebble, moving Car): -5 (or worse)
		\item High Gravity: -15
	\end{itemize}
\end{itemize}


\subsubsection{Believer}\label{Believer}
\textit{The character has strong beliefs, usually religious ones, that he is displaying openly.
He may openly wear insignia or iconography associated with his beliefs or actively preach about his beliefs to other persons.
Swaying the character to a new point of view or convince them to change a set course of action is generally a difficult prospect.}
\begin{itemize}
	\item \textbf{Aptitudes:} Willpower, Social
	\item \textbf{Common Attributes:} Willpower, Fellowship, intelligence
	\item \textbf{Skills:} Deny, Preach, Sanctify
	\item \textbf{Sample Consequences:} 
	\begin{itemize}
		\item Doubt: A sliver of doubt is enough to crack the firm belief. -5 on Believer checks for the remainder the scene.
		\item Shattered: The belief got shattered and makes it all but impossible to call on it for the remainder of the scene.
		\item Fanaticism: The Believer looses any sense of reality over the overpowering belief and cannot see reason. He is unable to use Rationalist skills for the remainder of the scene and his actions may overstep many boundaries he normally would not bend.
	\end{itemize}
	\item \textbf{Common Modifiers:}
	\begin{itemize}
		\item Corruption Points: -1 per CP the character or target has
		\item Betrayer Profession Rating: -3 per Betrayer Rating of the target
	\end{itemize}
\end{itemize}

\subsubsection{Betrayer}\label{Betrayer}
\textit{The character is focused on his own personal benefit in any given situation and values other people only for the benefit he can gain from using them.
The word of a Betrayer is worth as much as one might suspect and you better watch your goods when one is around.
Of course it is far from easy to see a Betrayer for who he really is through the maze of twisted lies, half-truth and even physical disguises a Betrayer might wear.
More often than not, a Betrayer has a likable face, his speech a honey-tinged sweet and his gestures inviting and calming - everything intended to lower suspicions and alertness.}
\begin{itemize}
	\item \textbf{Aptitudes:} Fellowship, Social
	\item \textbf{Common Attributes:}  Fellowship, Agility, Intelligence, WS
	\item \textbf{Skills:} Deceive, Disguise, Feint, Forgery, Selfishness, Sleight of Hand, Smuggle
	\item \textbf{Sample Consequences:} 
	\begin{itemize}
		\item Suspicious: Target becomes suspicious of the character. -5 for all future Betrayer checks. Stackable.
		\item Caught Red Handed: target knows the target to be a Betrayer. All Betrayer checks fail for the remainder of the scene. May restrict future interactions.
	\end{itemize}
	\item \textbf{Common Modifiers:}
	\begin{itemize}
		\item Target Believer Rating: +3 per Believer rating of the target
		\item Distracted Target: +15
	\end{itemize}
\end{itemize}

\subsubsection{Commander}\label{Commander}
\textit{The character is an imposing figure – not necessarily in a physical sense – but by the sheer air of authority and determination that is nearly palpable around him.
Commanders usually come in three variants: those who are deeply trusted by their comrades, those who are feared by them and those who seriously outsmart them.}
\begin{itemize}
	\item \textbf{Aptitudes:} Fellowship, Leadership
	\item \textbf{Common Attributes:} Fellowship, Willpower, Intelligence
	\item \textbf{Skills:} Battle Plan, Combat Sense, Inspire, Pull Rank, Rally, Steel
	\item \textbf{Sample Consequences:} 
	\begin{itemize}
		\item Grudging Obedience: The instruction does not go down well with the troops, lowering morale. -5 to all future Command skill checks for the remainder of the scene.
		\item Misinterpreted Command: The instruction was unclear or the receiver decides to misinterpret it. The target does not what it was instructed to do within the bounds of the command and circumstance.
		\item Disobedience: The target simply denies the instruction. This is practically mutiny! Cannot use Command skills on the target until the issue is resolved.
	\end{itemize}
	\item \textbf{Common Modifiers:}
	\begin{itemize}
		\item Profession Rating: the character adds +3 for each Profession point he has in: Spokesman, Believer, Enforcer or Rationalist (only the highest applies).
	\end{itemize}
\end{itemize}

\subsubsection{Enforcer}\label{Enforcer}
\textit{The character is a hardened man, used to get his way and his orders being followed. 
He is well used and most likely feared for his scrupulous ways and words let alone the punishment he has ever ready for those that deny him.
To stand up to an Enforcer is to face a brutal torturer of mind, body and soul. 
Most cannot withstand the unforgiving, razor sharp gaze from an Enforcer's cold eyes let alone keep their wits when he starts shouting.}
\begin{itemize}
	\item \textbf{Aptitudes:} strength, Social
	\item \textbf{Common Attributes:} Strength, Fellowship, Offense, Finesse, Intelligence
	\item \textbf{Skills:} Blackmail, Interrogate, Terrify, Torture, Warcry
	\item \textbf{Sample Consequences:} 
	\begin{itemize}
		\item Lingering Dislike: The target remembers the threats and actions against him by the character and will be harder to interact with in the future. -5 on any Spokesman, Commander skill checks until resolved.
		\item Growing Resolve: The target is  growing more resilient against the threats and actions of the Enforcer. -5 on Enforcer checks for the remainder of the scene.
		\item Empty Threats: The target has drawn the conclusion that the Enforcer's threats are null and empty. Future Enforcer checks fail until resolved.
	\end{itemize}
	\item \textbf{Common Modifiers:}
	\begin{itemize}
		\item Subject feels inferior: +10
		\item Subject feels superior: -10
		\item Subject fears worse if he gives in: -15 
	\end{itemize}
\end{itemize}

\subsubsection{Ghost}\label{Ghost}
\textit{The character is easy to oversee and forget.
He may have the habit of staying in the back and keeping quiet.
Maybe he has an unremarkable face and voice.
Or maybe he is just extremely able to blend in with his surrounding, be it a jungle or a crowded street.}
\begin{itemize}
	\item \textbf{Aptitudes:} Agility, Fieldcraft OR Finesse
	\item \textbf{Common Attributes:} Agility, Willpower, Perception
	\item \textbf{Skills:} Ghost Move, Hide, Shadowing, Vanish
	\item \textbf{Sample Consequences:} 
	\begin{itemize}
		\item What was that?: Target learns the coarse location of the character without knowing details. -5 on Ghost checks until resolved.
		\item Spotted: The target has spotted the character and is focused on him. All future Ghost checks fail until resolved (e.g. by using Distract/Spokesman or Maneuvering/Acrobat)
	\end{itemize}
	\item \textbf{Common Modifiers:}
	\begin{itemize}
		\item Profession Rating: +3 per Survivalist rating
		\item Paranoid target: -10
		\item Distracted Target: +10
	\end{itemize}
\end{itemize}

\subsubsection{Medic}\label{Medic}
\textit{The character is likely lean and with deep-set eyes from working countless over-hours in a hospital or apothecarium.
He will have delicate and dexterous fingers and sharp eyes.
More often than not, a Medic is quick to act for hesitation is ill at place when lives are at stake.}
\begin{itemize}
	\item \textbf{Aptitudes:} Intelligence OR Agility, Fieldcraft
	\item \textbf{Common Attributes:} Intelligence, Agility, Perception, Fellowship
	\item \textbf{Skills:} Diagnose, First Aid, Rehabilitate, Surgery
	\item \textbf{Sample Consequences:} 
	\begin{itemize}
		\item Clipped Nerve: target suffers -5 on all Agility tests for the remainder of the scene
		\item Bleeding: target suffers from a light bleeding
		\item Arterial Spray: the character is it by a spray of arterial blood from his patient and must pass an Endure check or suffer a strong immune reaction during the next scene, which causes a -10 penalty to all tests
		\item Non-Sterile Treatment: the patient must pass An Endure check or succumb to an infection in the next scene, which cause a -10 penalty on all checks. If not treated, this may prove fatal.
		\item Finger Cut: character cuts himself during treatment and suffers from a light bleeding.
	\end{itemize}
	\item \textbf{Common Modifiers:}
	\begin{itemize}
		\item Unknown/Altered biology: -30
		\item Struggling Patient: -15
		\item Lack of Assistance: -5
		\item Medical trained Assistant: +5
		\item Well-Equipped Hospital: +15
		\item Non-sterile environment: -10
	\end{itemize}
\end{itemize}



% section profession (end)

\section{Skills}% (fold)
\label{sec:skills}
Skills represent narrow, specialist activities and are the central resolution mechanic used in Grimdark. 
Every skill is associated with a Profession.
Thus, skills are usually written like this: Pretty Words (Spokesman).
That is the name of a skill is followed by the respective Profession in parenthesis.

\subsection{Subtle and Blatant Skills}
\label{skill_types}
Grimdark uses two different type of skills: Subtle and Blatant.
\begin{itemize}
	\item \textbf{Subtle Skills} are (somewhat) covert or low-profile in manner and draw considerable less attention than Blatant skills. 
	Subtle skills produce less severe effects on a failure or partial success but in turn are slightly less likely to succeed and loose a Draw during an opposed test.\\
	Examples: Pretty Words (Spokesman), Disguise (Betrayer), Balancing (Acrobat), Snap-Shot (Gunner), Fencing (Warrior)

	\item \textbf{Blatant Skills} are flashy, loud, extreme or all of them (and more) at once. 
	Using a blatant skill is nearly guaranteed to draw (potentially unwanted) attention.
	A Blatant skill will cause considerable effects on failure or partial success but has a higher chance to succeed and always win a Draw during an opposed test.\\
	Examples: Quarrel (Spokesman), Rally (Commander), Break (Athlete), Walking Fire (Gunner), Hammerblow (Warrior)
\end{itemize}
Some skills may be used in both Subtle or Blatant way, e.g. the Demolition (Tinkerer) skill. 
Discretely placing a bomb can be done in a subtle way, but using a charge to blow open a door and shake those within certainly can only be considered blatant.
The GM is the final arbiter when it comes to deciding whether a skill is used in a Blatant or Subtle way.
% section skills (end)