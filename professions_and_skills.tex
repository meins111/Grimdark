%!TEX root = ./Grimdark.tex
\chapter{Professions \& Specialties}
Grimdark's skill system is divided into so-called Professions, which group associated Specialties together. Raising a certain Specialty will also raise the profession rating, which can be used in place of any Specialty of a Profession. 
This represents the fact that specialists are also (somewhat) proficient in their general field of expertise.\\
A renown surgeon for cybernetic replacements is certainly also quite capable at first aid treatments after all, and a famous climber is quite likely a dangerous wrestler as well...
\begin{figure}[ht]
	\begin{DndReadAloud}
	{\large\textbf{Generalist or Specialist Path?}}

	The number of Specialties featured in Grimdark is quite large (and might even grow in your game!), so it is easy to feel overhwelmed.
	It is totally acceptable to completely set aside Specialties for your first (or any) character and only use the more general Professions instead! Grimdark calls this the \textbf{Generalist} path.
	Whenever the GameMaster calls for a Specialty check, he should also announce the respective Profession.
	If you choose to follow the Generalist path, you will just make the respective Profession check and you're good - the GM will apply the respective Modifiers of the Specialty and resolve the action.

	Should you prefer the delve deeper into Grimdark's skill system and manage each individual Specialty, you are set to follow the \textbf{Specialist} path.
	The Specialist path provides the option to fiddle with every last bit of your character, and can be seen as a tool for advanced players (or those who like a complex system) to tweak and customize their characters even further and make them truly unique.
	\end{DndReadAloud}
\end{figure}

\section{Professions} % (fold)
\label{sec:professions}
A Profession represent the generalist ability of a character in a certain broad area of expertise.
Each Profession covers a range of Specialties which are somewhat related.
In addition, they provide a fast and loose overview of the character.
Is he a brutal underhive ganger with high scores in the Betrayer, Athlete and Warrior professions?
Or an ancient tech priest with a more accumulated knowledge in his left (augmented) toe, than a handful of highborn nobles put together, reflected by his  Tinkerer, Technologist and Logic professions?
\begin{figure}[ht]
	\begin{DndSidebar}{Profession Example}
	\textit{Sister Aryanna of the Ordo Dialogis and her death-cult comrade Krill'Ta are attempting to infiltrate a gathering of noblewomen of the Hive World Olrabus, suspecting the machinations of a genestealer cult in its midst. As a distrustful guard is denying them entrance, a hasty debate between the two members of the Inquisitors retinue using Ordo hand signs erupts. Should Aryanna attempt to tell a convoluted (and utterly crafted) story to convince the guard - or should Krill'Ta attempt to seduce the guard with her death world charm?\\
	Both would be valid approaches, but seducing the guard may take Krill'Ta out of the next scene and leave the sister alone in the midst of the noblewomen. On the other hand, sister Aryanna isn't particular known for her abilities to lie - an ability which is far from well-perceived by her Ordo after all...\\}
	\end{DndSidebar}
\end{figure}

One central decision of Grimdark, contrary to many other game systems, is the fact that it does not assume that player characters or even a group of player characters, cover all or even most professions. 
There are many different ways to tackle any given situation scattered over the different professions that may appear redundant on a purely mechanical take.
But each profession has its very own take and spin on "their" solution to a problem, along with certain role-play implications, limitations and situational modifiers.
Thus a certain given situation may be much easier to solve by one profession than by another - yet they all have their uses, without forcing someone to break out of character to tackle a certain solution.

On a mechanical level, Professions work (mostly) like Specialtes, with the main difference is the fact that you cannot raise Professions directly and use a different modifier.
The roll modifier when using a Profession is determined by the current Profession rating (see below).

\begin{figure}[hb]
	\begin{DndReadAloud}
	\textbf{Design Decision: Professions}
	\begin{itemize}
		\item \textbf{Facilitate Player Creativity}: If a player has an idea, which is not covered by the existing Specialties, it is usually easy to discern the Profession to which the described action relates. Having the option to roll a Profession check in such a case, allows for great player creativity. GM and Players are encouraged to develop new Specialties for commonly used actions of players that are not foreseen by Grimdark.
		\item \textbf{Make Everyone a Generalist}: In many skill-based game systems, players may end up with specialist characters which are very proficient in skill A but absolutely untrained in skill B, although the two skills are actually quite closely related. This does not make a whole lot of sense, as every specialist is also necessarily a generalist in his field of purview - at least to some degree.
		\end{itemize}
	\end{DndReadAloud}
\end{figure}

\subsection{The Profession Rating}
Each profession has a certain Profession rating, representing the total ability of the character in that Profession. 
This rating \textbf{increases by 2 for each new Specialty level learned} in an associated Specialty.

\subsection{The Profession Check}
A Profession check works just like a Specialty check with a single difference.
Instead of using the respective Specialty modifier, it uses the Profession Rating.
Everything else works the same, including the used attributes and situational modifiers of the respective Specialty.\\
With the GM's accordance, a character may even attempt a Profession check if he does not have the respective Profession (and thus a Profession ratimg of 0).
In such a case, the player takes a -10 modifier on the roll and he may only achieve a Partial Sucess - never a Full Success.

\begin{figure}[hb]
	\begin{DndSidebar}{Profession Check Example}
	\textit{Inquisitor Otto Brecht has been trapped into a hive corridor with all doors locked. The straight forward Inquisitors decides to try and force open one of the doors with his own strength, hoping to still remain his covert approach that way. If that fails, he can still blast open the doors with a grenade after all ...\\
	Brecht is not particularly skilled at breaking things - he relies on the brute Nic'Or for such things - but he is keeping himself in top physical condition and is tough as nail. He should have a good enough chance to break open the doors...}\\
	That is, Brecht does not have the Break Specialty of the Athlete Profession, but he does have two ranks each in the Triathlete and Endure Specialties of that Profession. Thus, his Athlete Profession rating is: $2*4=8$.
		\begin{itemize}
		\item \textbf{Specialty / Profession:} Break / Athlete
		\item \textbf{Specialty Modifier:} 0 (untrained => use Profession)
		\item \textbf{Profession Rating:} 8
		\item \textbf{Attributes:} Power 37, Toughness 44
		\item \textbf{Modifiers:} Locked Door (-10)
		\item \textbf{Target Numbers:} 35 (37+8-10), 42 (44+8-10)
		\item \textbf{Rolls:} 77 (no pass), 43 (no pass)
		\item \textbf{Result:} Failure
	\end{itemize}
	\textit{But the door won't budge. Going loud it is then. So be it. He never liked sneaking around anyway...}
	\end{DndSidebar}
\end{figure}

\subsection{Synergies}\label{Synergy}
In many situations, a character may benefit from being good in another Profession beside the one he is currently using.
When the player can make a sound argument why in a given situation such a Synergy applies, he may add half the Profession rating of the supporting Profession as a modifier on top of his check. 
As usual, the GM is the final arbiter whether a Synergy is applicable or not, but he is encouraged to allow most such calls to encourage player creativity.
\begin{DndSidebar}{Synergy Example}
	\textit{Nic'Or the tribal warrior of the backwater planet Or, is currently having a bit of a trouble. He was expected to secure the roof of a hive spire while his Inquisitor was infiltrating it through the air vents. But now, the roof is quickly filling with security personal and he has strict order to not be detected. His only hope of staying hidden, is if he can quickly get off the roof - that is he has to jump to one of the neighboring rooftops. Nic'Or is a strong warrior with a body practically overflowing with muscle - but he is not an acrobat like some of the Inquisitors other retinue members. Despite being unsure whether he will manage the far leap, he has no other choice. While building momentum, he offers his soul to the Emper'Or and leaps...}\\
	Nic'Or is not skilled in the Acrobat Profession, thus he takes a -10 penalty when using the Leap/Acrobat Specialty. Nic'Or's player does point out however, that the great athletic Nic'Or should benefit from a Synergy from the Athlete Profession, since making a far jump does need quite a bit of strength alongside the right technique and timing after all.\\
	The GM and the other players agree. Nic'Or's player can thus add half of Nic'Or's (considerable) Athlete Profession Rating to the check and now has a far greater chance of succeeding.
\end{DndSidebar}


\subsection{The Professions}
Grimdark uses the following professions:
\begin{itemize}
	\item \textbf{\nameref{Acrobat}:} covers Specialties that require a high level of body control, tension and swiftness.
	\item \textbf{\nameref{Athlete}:} covers Specialties that require a high level of physical strength and endurance.
	\item \textbf{\nameref{Believer}:} covers Specialties that require a deep conviction in a belief.
	\item \textbf{\nameref{Betrayer}:} covers Specialties that deceive others or break common law.
	\item \textbf{\nameref{Commander}:} covers Specialties that lead others by example, pressure or wit.
	\item \textbf{\nameref{Enforcer}:} covers Specialties that cow others into submission or fear.
	\item \textbf{\nameref{Ghost}:} covers Specialties that allow the character to stay hidden.
	\item \textbf{\nameref{Gunman}:} covers Specialties of ranged weapon fighting.
	\item \textbf{\nameref{Loremaster}:} covers a wide range of Knowledge Specialties.
	\item \textbf{\nameref{Medic}:} covers Specialties needed to keep others alive.
	\item \textbf{\nameref{Navigator}:} covers Specialties that prevent you from getting lost.
	\item \textbf{\nameref{Observer}:} covers Specialties that require a high level of alertness and perceptiveness.
	\item \textbf{\nameref{Operator}:} covers Specialties that allow the operation of vehicles and its functions.
	%\item \textbf{\nameref{Psyker}:} covers the arcane Specialties only open to the witch.
	\item \textbf{\nameref{Rationalist}:} covers Specialties that require a high level of intelligence and clear thought.
	\item \textbf{\nameref{Spokesman}:} covers Specialties that rely on communicating with others.
	\item \textbf{\nameref{Survivalist}:} covers Specialties that allow characters to survive in any environment on their own.
	\item \textbf{\nameref{Technologist}:} covers Specialties that allow the proper use, understanding and even crafting of advanced tech.
	\item \textbf{\nameref{Tinkerer}:} covers Specialties that allow the proper use of common and low tech.
	\item \textbf{\nameref{Trader}:} covers Specialties related to money, property and value.
	\item \textbf{\nameref{Warrior}:} covers Specialties of close-quarter fighting.
\end{itemize}

Below follows the entries of the different Professions. 
All use a common layout, as follows:
\begin{itemize}
	\item First in italics, a short description of the respective Profession is given, focusing on common appearance and behavior patterns of characters of that Profession.
	These should not be taken as strict guides but as more as helpful guidelines when creating and playing a character.
	\item Aptitudes: the here listed Aptitudes regulate how difficult it is for a given character to learn this Profession.
	\item Common Attributes: this ordered list provide an overview over the Attributes used by the Specialties associated with the Profession. 
	The further back an Attribute is listed, the fewer Specialties make use of it.
	This list is intended to be a guide for players following the Generalist path, so they know what attributes their characters will likely need when using the Specialty.
	\item Tags: A list of \nameref{Tags} the Specialties of this Profession cover.
	\item Specialties: An alphabetical list of Specialties associated with this Profession.
	Players following the Generalist Path may use this list as a reference to spark their creativity and problem solving approaches.
	\item Synergies: A list of \underline{possible} Synergies, that is other Professions that may be commonly used as Synergy candidates for the Profession. Keep in mind that the listed ones are only proposals and even if none is listed, a good argument may still convince the GM to allow a Synergy!
	\item Sample Consequences: a list of examples of what might happen when a character fails or get a partial success with consequences on a task using this Profession.
	\item Common Modifiers: a list of common modifiers that might situationally apply when using this Profession.
\end{itemize}

\subsubsection{Acrobat}\label{Acrobat}
\textit{The character shows a high level of body control and tension. 
His movements are fluid, every step measured, every gesture a study in precision.
He is likely to possess quite some strength without appearing swollen with muscle and may appear lean or even slim without being sunken or weak.
The character is likely to be considered handsome or even beautiful depending on the local beauty ideals and other people might react accordingly.}
\begin{itemize}
	\item \textbf{Aptitudes:} Finesse, Subtlety
	\item \textbf{Common Attributes:} Finesse, Power, Toughness, Will
	\item \textbf{\nameref{Tags}:} SOL, PD, SUP	
	\item \textbf{Specialties:} Balancing, Contortionist, Dancing, Hard Target, Leap, Maneuvering
	\item \textbf{Synergies:} Athlete
	\item \textbf{Sample Consequences:} 
	\begin{itemize}
		\item Twisted Ankle: Cannot run or charge for the remainder of the scene.
		\item Snapped Tendon: -5 on any Finesse based check until healed.
		\item Bruise: 2d5 points of Resilience damage.
		\item Trip: Fall prone.
		\item Wide Open: Enemies gain +10 to attack the character until his next turn.
	\end{itemize}
	\item \textbf{Common Modifiers:}
	\begin{itemize}
		\item High Gravity: -15
		\item Zero Gravity: -10
		\item No Armor worn: +15
		\item Bulky Weapon: -5
		\item Slippery Ground (Water, Ice, Oil): -5 (or worse)
	\end{itemize}
\end{itemize}

\subsubsection{Athlete}\label{Athlete}
\textit{The character is visibly strong and tough.
He likely has great muscle mass and little to no fat deposits.
Most people will expect the character to be a good warrior - which may influence the way they perceive the character and react to him.
He may be physically intimidating to less strong characters even without meaning to.}
\begin{itemize}
	\item \textbf{Aptitudes:} Power OR Toughness, Offense
	\item \textbf{Common Attributes:} Power, Toughness, Will, Weapon Skill, 
	\item \textbf{\nameref{Tags}:} PA, PD, SOL
	\item \textbf{Specialties:} Block, Brawl, Break, Endure, Heave, Throw, Triathlete
	\item \textbf{Synergies:} Acrobat
	\item \textbf{Sample Consequences:} 
	\begin{itemize}
		\item Torn Muscle: Cannot perform any Athlete skill checks until healed.
		\item Bruise: 2d5 Resilience damage
		\item Fractured Bone: Increases resilience damage taken by 2 until healed. Cumulative.
		\item Push-Back: The character is pushed back by 1d5 meters.
		\item Contusion: -5 on any Power based check.
	\end{itemize}
	\item \textbf{Common Modifiers:}
	\begin{itemize}
		\item Solid Ground: +5
		\item Unsteady Ground (Pebble, moving Car): -5 (or worse)
		\item High Gravity: -15
	\end{itemize}
\end{itemize}


\subsubsection{Believer}\label{Believer}
\textit{The character has strong beliefs, usually religious ones, that he is displaying openly.
He may openly wear insignia or iconography associated with his beliefs or actively preach about his beliefs to other persons.
Swaying the character to a new point of view or convince them to change a set course of action is generally a difficult prospect.}
\begin{itemize}
	\item \textbf{Aptitudes:} Will OR Charisma, Social
	\item \textbf{Common Attributes:} Will, Charisma, Smarts
	\item \textbf{\nameref{Tags}:} MA, MD, SUP, SOL
	\item \textbf{Specialties:} Deny, Inspire, Preach, Sanctify
	\item \textbf{Synergies:} Spokesman, Loremaster
	\item \textbf{Sample Consequences:} 
	\begin{itemize}
		\item Doubt: A sliver of doubt is enough to crack the firm belief. -5 on Believer checks for the remainder the scene.
		\item Shattered: The belief got shattered and makes it all but impossible to call on it for the remainder of the scene.
		\item Fanaticism: The Believer looses any sense of reality over the overpowering belief and cannot see reason. He is unable to use Rationalist skills for the remainder of the scene and his actions may overstep many boundaries he normally would not bend.
	\end{itemize}
	\item \textbf{Common Modifiers:}
	\begin{itemize}
		\item Corruption Points: -1 per CP the character or target has
		\item Betrayer Profession Rating: -1 per Betrayer Rating of the target
	\end{itemize}
\end{itemize}

\subsubsection{Betrayer}\label{Betrayer}
\textit{The character is focused on his own personal benefit in any given situation and values other people only for the benefit he can gain from using them.
The word of a Betrayer is worth as much as one might suspect and you better watch your goods when one is around.
Of course it is far from easy to see a Betrayer for who he really is through the maze of twisted lies, half-truth and even physical disguises a Betrayer might wear.
More often than not, a Betrayer has a likable face, his speech a honey-tinged sweet and his gestures inviting and calming - everything intended to lower suspicions and alertness.}
\begin{itemize}
	\item \textbf{Aptitudes:} Charisma OR Finesse, Social
	\item \textbf{Common Attributes:}  Charisma, Finesse, Smarts, WS
	\item \textbf{\nameref{Tags}:} MA, MD, SOL, SUP
	\item \textbf{Specialties:} Deceive, Disguise, Feint, Forgery, Selfishness, Sleight of Hand, Smuggle
	\item \textbf{Synergies:} Spokesman, Ghost, Loremaster
	\item \textbf{Sample Consequences:} 
	\begin{itemize}
		\item Suspicious: Target becomes suspicious of the character. -5 for all future Betrayer checks. Stackable.
		\item Caught Red Handed: target knows the target to be a Betrayer. All Betrayer checks fail for the remainder of the scene. May restrict future interactions.
	\end{itemize}
	\item \textbf{Common Modifiers:}
	\begin{itemize}
		\item Target Believer Rating: +3 per Believer Rating of the target when using the targets Believes against it
		\item Distracted Target: +15
	\end{itemize}
\end{itemize}

\subsubsection{Commander}\label{Commander}
\textit{The character is an imposing figure – not necessarily in a physical sense – but by the sheer air of authority and determination that is nearly palpable around him.
Commanders usually come in three variants: those who are deeply trusted by their comrades, those who are feared by them and those who seriously outSmarts them.}
\begin{itemize}
	\item \textbf{Aptitudes:} Charisma, Leadership
	\item \textbf{Common Attributes:} Charisma, Will, Smarts
	\item \textbf{\nameref{Tags}:} MA, SUP
	\item \textbf{Specialties:} Battle Plan, Combat Sense, Pull Rank, Rally, Steel
	\item \textbf{Synergies:} Spokesman, Enforcer, Believer, Betrayer, Rationalist, Loremaster
	\item \textbf{Sample Consequences:} 
	\begin{itemize}
		\item Grudging Obedience: The instruction does not go down well with the troops, lowering morale. -5 to all future Command skill checks for the remainder of the scene.
		\item Misinterpreted Command: The instruction was unclear or the receiver decides to misinterpret it. The target does not what it was instructed to do within the bounds of the command and circumstance.
		\item Disobedience: The target simply denies the instruction. This is practically mutiny! Cannot use Command Specialties on the target until the issue is resolved.
	\end{itemize}
	\item \textbf{Common Modifiers:}
	\begin{itemize}
		\item Dangerous Command: -5 (or worse)
		\item Simple Command: +5 (or better)
		\item Complex Command: -5 (or worse)
		\item Target is Superior: -10
		\item Target is Subordinate: +10
	\end{itemize}
\end{itemize}

\subsubsection{Enforcer}\label{Enforcer}
\textit{The character is a hardened man, used to get his way and his orders being followed. 
He is well used and most likely feared for his scrupulous ways and words let alone the punishment he has ever ready for those that deny him.
To stand up to an Enforcer is to face a brutal torturer of mind, body and soul. 
Most cannot withstand the unforgiving, razor sharp gaze from an Enforcer's cold eyes let alone keep their wits when he starts shouting.}
\begin{itemize}
	\item \textbf{Aptitudes:} Power, Social
	\item \textbf{Common Attributes:} Power, Charisma, Offense, Finesse, Smarts
	\item \textbf{\nameref{Tags}:} MA, SOL, SUP
	\item \textbf{Specialties:} Blackmail, Interrogate, Terrify, Torture, Warcry
	\item \textbf{Synergies:} Spokesman, Athlete, Betrayer, Trader, Loremaster
	\item \textbf{Sample Consequences:} 
	\begin{itemize}
		\item Lingering Dislike: The target remembers the threats and actions against him by the character and will be harder to interact with in the future. -5 on any Spokesman, Commander skill checks until resolved.
		\item Growing Resolve: The target is  growing more resilient against the threats and actions of the Enforcer. -5 on Enforcer checks for the remainder of the scene.
		\item Empty Threats: The target has drawn the conclusion that the Enforcer's threats are null and empty. Future Enforcer checks fail until resolved.
	\end{itemize}
	\item \textbf{Common Modifiers:}
	\begin{itemize}
		\item Subject feels inferior: +10
		\item Subject feels superior: -10
		\item Subject fears worse if he gives in: -15 
	\end{itemize}
\end{itemize}

\subsubsection{Ghost}\label{Ghost}
\textit{The character is easy to oversee and forget.
He may have the habit of staying in the back and keeping quiet.
Maybe he has an unremarkable face and voice.
Or maybe he is just extremely able to blend in with his surrounding, be it a jungle or a crowded street.}
\begin{itemize}
	\item \textbf{Aptitudes:} Finesse, Subtlety
	\item \textbf{Common Attributes:} Finesse, Will, Instinct
	\item \textbf{\nameref{Tags}:} SOL
	\item \textbf{Specialties:} Ghost Move, Hide, Shadowing, Vanish
	\item \textbf{Synergies:} Observer, Survivalist, Spokesman, Betrayer, Loremaster
	\item \textbf{Sample Consequences:} 
	\begin{itemize}
		\item What was that?: Target learns the coarse location of the character without knowing details. -5 on Ghost checks until resolved.
		\item Spotted: The target has spotted the character and is focused on him. All future Ghost checks fail until resolved (e.g. by using Distract/Spokesman or Maneuvering/Acrobat)
	\end{itemize}
	\item \textbf{Common Modifiers:}
	\begin{itemize}
		\item Paranoid target: -10
		\item Distracted Target: +10
	\end{itemize}
\end{itemize}

\subsubsection{Gunman}\label{Gunman}
\textit{The character is well versed in the usage and basic maintenance of firearms of different kind.
It is very likely that he is carrying one or multiple guns - whether openly or covert - and would feel positively naked when not.}
\begin{itemize}
	\item \textbf{Aptitudes:} Finesse, Offense
	\item \textbf{Common Attributes:} Power, Finesse, Instinct, Will
	\item \textbf{\nameref{Tags}:} PA
	\item \textbf{Specialties:} Central Mass, Sniping, Sweeping Fire, Snap Shot, Walking Fire, Trick Shot
	\item \textbf{Synergies:} Rationalist, Observer
	\item \textbf{Sample Consequences:}
	\begin{itemize}
		\item Glancing Hit: not a miss but also not a direct hit - target take half damage.
		\item Malfunction: The weapon has a malfunction - you should take better care of it! Roll on the Malfunction Table.
	\end{itemize}
	\item \textbf{Common Modifiers:}
	\begin{itemize}
		\item Difficult Weather: -10
		\item Target is moving quickly: -10 (or worse)
		\item Target Size
	\end{itemize}
\end{itemize}

\subsubsection{Loremaster}\label{Loremaster}
\textit{Nothing easier than to recognize a Loremaster.
Just make a false statement about some incredible niche tidbit and watch who is getting all twitchy and finally cracks up and starts berating and correcting your false claims for the next hour or five - there: easy.}
\begin{DndReadAloud}
\textbf{\underline{Rule Exception:} Using the Loremaster Profession Rating}\\
As an exception to the normal rule, you cannot use your Loremaster Profession Rating instead of Loremaster Specialties, which you did not learn yet. You can still use your Profession Rating instead of the Specialty modifier of a learned Specialty though.\\
It simply does not make any sense for a learning scribe of the Administratum (who knows a lot about the Imperium of Man) to be an expert in Occult things or Xenology after all...
\end{DndReadAloud}
\begin{itemize}
	\item \textbf{Aptitudes:} Smarts, Knowledge
	\item \textbf{Common Attributes:} Smarts, Instinct, Will
	\item \textbf{\nameref{Tags}:} SOL, SUP
	\item \textbf{Specialties:} Classic Education, Imperium of Man, Occult, Shadow Organizations, Tactica Imperialis, Ways of Mars, Xenology
	\item \textbf{Synergies:} Rationalist (Classic Education), Psyker (Occult), Commander (Tactica Imperialis), Technologist (Ways of Mars), ...
	\item \textbf{Sample Consequences:} 
	\begin{itemize}
		\item Wrong Recollection: The character recalls a false information and is convinced of its truth until convinced or proven otherwise.
		\item Partial Recollection: The character recalls most of the wanted data, but for some (potentially critical) bit.
		\item It was the other Way Round! The character mixed two things up. 43 becomes 34. Left corridor becomes right corridor. Red cable becomes the green.
	\end{itemize}
	\item \textbf{Common Modifiers:}
	\begin{itemize}
		\item Common Lore: +10
		\item Uncommon Lore: -5
		\item Forbidden Lore: -15
	\end{itemize}
\end{itemize}

\subsubsection{Medic}\label{Medic}
\textit{The character is likely lean and may have rings under his eyes from working countless over-hours in a hospital of sorts.
Most Medics have delicate and dexterous fingers coupled with sharp eyes and a bright mind.
More often than not, a Medic is quick to act for hesitation is ill at place when lives are at stake.
And then there is that aura of competence and calm around them, that marks them out more than any of the above.}
\begin{itemize}
	\item \textbf{Aptitudes:} Smarts OR Finesse, Fieldcraft
	\item \textbf{Common Attributes:} Smarts, Finesse, Instinct, Charisma
	\item \textbf{\nameref{Tags}:} SOL, SUP
	\item \textbf{Specialties:} Diagnose, First Aid, Medical Crafting, Rehabilitate, Surgery
	\item \textbf{Synergies:} Rationalist, Loremaster, Spokesman
	\item \textbf{Sample Consequences:} 
	\begin{itemize}
		\item Clipped Nerve: target suffers -5 on all Finesse tests for the remainder of the scene
		\item Bleeding: target suffers from a light bleeding
		\item Arterial Spray: the character is it by a spray of arterial blood from his patient and must pass an Endure check or suffer a strong immune reaction during the next scene, which causes a -10 penalty to all tests
		\item Non-Sterile Treatment: the patient must pass An Endure check or succumb to an infection in the next scene, which cause a -10 penalty on all checks. If not treated, this may prove fatal.
		\item Finger Cut: character cuts himself during treatment and suffers from a light bleeding.
	\end{itemize}
	\item \textbf{Common Modifiers:}
	\begin{itemize}
		\item Unknown/Altered biology: -30
		\item Struggling Patient: -15
		\item Lack of Assistance: -5
		\item Medical trained Assistant: +5
		\item Well-Equipped Hospital: +15
		\item Non-sterile environment: -10
	\end{itemize}
\end{itemize}

\subsubsection{Navigator}\label{Navigator}
\textit{Knowing your way around places others would call labyrinthine is the characters second nature.}
\begin{itemize}
	\item \textbf{Aptitudes:} Smarts, Fieldcraft
	\item \textbf{Common Attributes:} Smarts, Instinct, Will
	\item \textbf{\nameref{Tags}:} SOL, SUP
	\item \textbf{Specialties:} Enclosed, Surface, Void, Warp
	\item \textbf{Synergies:} Rationalist, Observer, Loremaster
	\item \textbf{Sample Consequences:} 
	\begin{itemize}
		\item Detour: The character does not find the best possible path, causing the voyage to take some additional time
		\item Dangerous Path: The character follows a dangerous path. There might be environmental hazard ahead or maybe an unfriendly encounter.
		\item Difficult Path: The route the character leads to, has a couple of difficulties, which may require additional checks fitting to the environment.
		\item Lost: The character has completely lost orientation and may end in a totally different location than he wanted to. 
	\end{itemize}
	\item \textbf{Common Modifiers:}
	\begin{itemize}
 		\item Alien Environment: -15
 		\item Familiar Environment: +10
 		\item Difficult Weather: -10
 		\item Lack of a Map: -10
 		\item Coarse Map: +5
 		\item Good Map: +15
	\end{itemize}
\end{itemize}

 \subsubsection{Observer}\label{Observer}
 \textit{An old saying goes as follow: it takes a good Observer to notice an Observer.
Keen senses, long experience and good instincts come together to form an Observer.
You may notice that first thing he does when walking into a room is looking around with practiced eye movements or that small smirk on his face when he sees through one of your bluffs - that is if you're lucky and he is not good at hiding those minute movements.}
 \begin{itemize}
 	\item \textbf{Aptitudes:} Instincts, Subtlety
 	\item \textbf{Common Attributes:} Instincts, Smarts, Finesse
 	\item \textbf{\nameref{Tags}:} MD, SOL, SUP
 	\item \textbf{Specialties:} Lip Reading, Sixth Sense, Search, Scrutiny
 	\item \textbf{Synergies:} Loremaster, Rationalist
 	\item \textbf{Sample Consequences:} 
 	\begin{itemize}
 		\item False Negative: you have gotten accustomed to a sound or sight you better not have. -5 Observer skills while in this environment.
 		\item False Positive: you thought to have notices something and, following your instincts, act accordingly.
 	\end{itemize}
 	\item \textbf{Common Modifiers:}
 	\begin{itemize}
 		\item Intense Combat: -5
 		\item Alien Environment/Opponent: -15
 		\item Familiar Environment/Opponent: +10
 	\end{itemize}
 \end{itemize}

 \subsubsection{Operator}\label{Operator}
 \textit{Daring and skilled drivers, pilots and helmsman more often than not have a tendency for bragging, swearing and partying.
 Often enough they regularly bend the rules and expectations set for them by their organizations - but the really good ones are those that are allowed such debauchery for their immense skill is most valuable indeed.}
 \begin{itemize}
 	\item \textbf{Aptitudes:} Finesse OR Instinct, Offense OR Tech
 	\item \textbf{Common Attributes:} Finesse, Instinct, Will
 	\item \textbf{\nameref{Tags}:} SOL, SUP
 	\item \textbf{Specialties:} Aircraft, Personal, Tracked, Walker, Wheeled, Void, Xeno
 	\item \textbf{Synergies:} Technologist, Tinkerer, Loremaster
 	\item \textbf{Sample Consequences:} 
 	\begin{itemize}
 		\item Scratched the Paint: the character calculated to generous and the vehicle is having a light very light crash with whatever the character tried to avoid. The craft must pass a Structural Integrity test or one of its external components gets damaged.
 		\item Safety Protocols Activated: Now the character has gone too far and tasked the vehicle a bit too much. Some safety feature has activated to prevent permanent damage and is reducing some of its features for the time being. -5 to future Operator skills for the duration of the scene.
 		\item Lost Control: the character momentarily looses control over his vehicle. Depending on the situation this might cause anything from a short moment of panic all the way up to a (dangerous) crash.
 		\item Crash: the character did not manage to evade a potential crash and the vehicle is hitting an obstacle at speed. Oh-Oh...
 	\end{itemize}
 	\item \textbf{Common Modifiers:}
 	\begin{itemize}
 		\item Alien Craft: -20
 		\item Familiar Craft: +5
 		\item Personalized Craft: +10
 		\item Difficult Environment: -5 (or worse)
 		\item Easy environment: +5 (or better)
 		\item Crammed Surrounding: -10 (or worse)
 	\end{itemize}
 \end{itemize}

 \subsubsection{Rationalist}\label{Rationalist}
 \textit{True Rationalists are a rare breed in the grimdark future but yet they exist and try their utmost to unravel the mysteries of the universe with but their sharp minds, pointed questions and clever methods.
 More often than not, a Rationalist will show great interest in the "why" of any given situation.
 Others love to berate these around them and take any chance to share their knowledge and understandings of literally anything thy might encounter.}
 \begin{itemize}
 	\item \textbf{Aptitudes:} Smarts, Knowledge
 	\item \textbf{Common Attributes:} Smarts, Instinct, Will
 	\item \textbf{\nameref{Tags}:} MD, SOL, SUP
 	\item \textbf{Specialties:} Calculus, Cipher, Linguistic, Rationality
 	\item \textbf{Synergies:} Loremaster
 	\item \textbf{Sample Consequences:} 
 	\begin{itemize}
 		\item Miscalculation: The predictions are wrong (probably due to mistaken assumptions). Yet, the character is dead set on its truthfulness and will act accordingly.
 		\item Confused: The character does not manage to explain the problem with is knowledge and the methods of rationality. He will continue to think about the matter and will thus be distracted. -5 on any Smarts or Instinct based checks for the remainder of the scene.
 	\end{itemize}
 	\item \textbf{Common Modifiers:}
 	\begin{itemize}
 		\item Unnatural Influence: -20
 		\item Requiring Obscure Knowledge: -10
 	\end{itemize}
 \end{itemize}

\subsubsection{Spokesman}\label{Spokesman}
\textit{Every word coming from your mouth is well balanced, well pronounced ... and it just feels right.
Often enough, the smooth talk is coupled with a spotless appearing starting with a nice and well fitting dress or suite, creating an utterly respectable combination most people are more than ready to listen to and accept prompts, hints, recommendations or even well-meaning criticism.}
\begin{itemize}
	\item \textbf{Aptitudes:} Charisma, Social
	\item \textbf{Common Attributes:} Charisma, Instinct, Smarts
	\item \textbf{\nameref{Tags}:} MA, MD, SOL, SUP
	\item \textbf{Specialties:} Distract, Inquiry, Pretty Words, Quarrel, Seduce
	\item \textbf{Synergies:} Loremaster
	\item \textbf{Sample Consequences:} 
	\begin{itemize}
		\item Gone too Far: the character has overstepped one of the many invisible borders that riddle the social environment. The target will likely cancel the interaction in a fitting way - which may range from a politely end of the small talk all the way to flying fists.
		\item Ill Chosen Word: Well that was not all that nice, was it? -5 for Spokesman checks for the duration of the scene.
		\item Ignored: The target is done with the character and does not care for his words any more, regardless how pretty they may be. All Spokesman checks fail for the rest of the scene.
		\item Suspicious: The target doubts that the motivation of the character are in his favor. -10 on Spokesman, Betrayer and Believer checks for the remainder of the scene.
	\end{itemize}
	\item \textbf{Common Modifiers:}
	\begin{itemize}
		\item Target is Attracted: +5 (or more)
		\item Target is bored: +5
		\item Target is hostile: -15
		\item Target is busy: -5
	\end{itemize}
\end{itemize}

\subsubsection{Survivalist}\label{Survivalist}
\textit{Noticing a Survivalist in a civilized environment is not easy.
Maybe he has the habit of checking his surrounding nearly as careful as an Observer or has the habit of carying around lots of baggage - just in case.
But once you are in a desolate place, far from the benefits of civilization, there is no denying the aura of calm and competence of a true Survivalist.
While you may worry about the dangers of the wilderness (which may also apply to what is left in the ruins of a civilized place), including thirst, hunger and wild animals, the Survivalist just seems ... home at last.}
\begin{itemize}
	\item \textbf{Aptitudes:} Instinct, Fieldcraft
	\item \textbf{Common Attributes:} Instinct, Finesse, Smarts, Will
	\item \textbf{\nameref{Tags}:} SOL, SUP
	\item \textbf{Specialties:} Cooking, Foraging, Primitive Crafting, Taming, Zoology 	
	\item \textbf{Synergies:} Ghost, Observer, Rationalist, Loremaster
	\item \textbf{Sample Consequences:} 
	\begin{itemize}
		\item Polluted Food/Water: the character was not careful enough and foraged or cooked something that he would better not. Everyone who ate or drank from it must pass and Endure/Athlete check or get sick.
	\end{itemize}
	\item \textbf{Common Modifiers:}
	\begin{itemize}
		\item Alien Environment: -15
		\item Familiar Environment: +10
		\item Basic Tools: +5
		\item Lack of Tools: -10
	\end{itemize}
\end{itemize}

\subsubsection{Technologist}\label{Technologist}
\textit{The marvel of Technology are beyond the grasp of the vast majority of humanity.
But not for a Technologist.
His is the task of maintaining, repairing and even creating the advanced technology without which humanity would be all but doomed in the grimdark future.
Many feel such a deep connection to technology, that they surround themselves with it permanently - in the shape of mechanical augmentations and a flurry of gadgets, some hardwired to their body some only a swift grip away.}
\begin{itemize}
	\item \textbf{Aptitudes:} Smarts, Tech
	\item \textbf{Common Attributes:} Smarts, Finesse, Instinct, Will 
	\item \textbf{\nameref{Tags}:} SOL
	\item \textbf{Specialties:} Advanced Crafting, Communion, Hacking, Rites of Activation, Rites of Maintenance
	\item \textbf{Synergies:} Rationalist, Tinkerer, Loremaster
	\item \textbf{Sample Consequences:} 
	\begin{itemize}
		\item Angered Machine Spirits: The careless action of the character angered the machine spirit. In addition to potential acts of unrest, every Technologist or Tinkerer actions with the target suffers a -10 penalty for the remainder of the scene.
		\item Never Fix a Running System: The target is now (partially) broken and in a worse state then before. Its functions is limited until repaired - properly this time. If the character attempts that repair, he suffers a -10 penalty.
		\item How does this even work: The character is confused by the inner workings of the target. He may use it (unless its broken) but he cannot attempt modifications or repairs until he figures it out, usually by passing a Calculus/Rationalist check.
	\end{itemize}
	\item \textbf{Common Modifiers:}
	\begin{itemize}
		\item Alien Tech: -15
		\item Archeotech: -15
		\item Common Tech: +10
		\item Complex Tech: -5
		\item Friendly Spirits: +5 (or better)
		\item Unfriendly Spirits: -5 (or worse)
	\end{itemize}
\end{itemize}

\subsubsection{Tinkerer}\label{Tinkerer}
\textit{Lacking thorough understanding of technology did never hinder people from using it.
A Tinkerer is just one of those - he may use tech on a daily basis, maybe even some rather advanced tech, but all he knows is how to use it, in the way he was once taught to - or even figured it all by out himself.
Many Tinkerers proudly show off their skills, hoping to impress more careful people around them.
And despite this being close to (or even definite) tech heresy, many Tinkerers fiddle around with common day or even advanced tech in their free time.}
\begin{itemize}
	\item \textbf{Aptitudes:} Finesse, Fieldcraft
	\item \textbf{Common Attributes:} Finesse, Smarts, Instinct, Will
	\item \textbf{\nameref{Tags}:} PA, SOL
	\item \textbf{Specialties:} Common Crafting, Demolition, Lockpicking, Technomat
	\item \textbf{Synergies:} Rationalist, Loremaster
	\item \textbf{Sample Consequences:} 
	\begin{itemize}
		\item Angered Machine Spirits: The careless action of the character angered the machine spirit. In addition to potential acts of unrest, every Technologist or Tinkerer actions with the target suffers a -10 penalty for the remainder of the scene.
		\item Never Fix a Running System: The target is now (partially) broken and in a worse state then before. Its functions is limited until repaired - properly this time. If the character attempts that repair, he suffers a -10 penalty.
		\item This is some serious Tech: The character has no idea how to use the target piece of technology. He cannot use his Tinkerer ability to handle this machine until he is shown or learns how to do so.
	\end{itemize}
	\item \textbf{Common Modifiers:}
	\begin{itemize}
		\item Alien Tech: -20
		\item Archeotech: -20
		\item Common Tech: +10
		\item Simple Tech: +15
		\item Complex Tech: -10
		\item Friendly Spirits: +5 (or better)
		\item Unfriendly Spirits: -5 (or worse)
	\end{itemize}
\end{itemize}

\subsubsection{Trader}\label{Trader}
\textit{For a Trader, everything is about money, influence, debt, credit and power.
Usually he will show off his wealth by wearing expensive clothing and jewelry.
When he has visitors, he will ensure they are led through rooms filled with art, expensive items and other luxuries.}
\begin{itemize}
	\item \textbf{Aptitudes:} Smarts, Knowledge OR Social
	\item \textbf{Common Attributes:} Smarts, Instinct, Charisma; Power
	\item \textbf{\nameref{Tags}:} MA, SOL
	\item \textbf{Specialties:} Bookkeeping, Bribe, Estimate, Haggle
	\item \textbf{Synergies:} Spokesman, Betrayer, Rationalist, Loremaster
	\item \textbf{Sample Consequences:} 
	\begin{itemize}	
		\item Underestimated: The character judged the price of an item or service too low. He may sell something too cheap or may have a hard time buying it for the assumed price.
		\item Overestimated: The character judged the price of an item or service too high. He may struggle to find someone to sell it to or spend too much when buying something.
	\end{itemize}
	\item \textbf{Common Modifiers:}
	\begin{itemize}
		\item Alien Economy: -15
		\item Foreign Economy: -5
		\item Familiar Economy: +5 
	\end{itemize}
\end{itemize}

\subsubsection{Warrior}\label{Warrior}
\textit{As a master of close-combat fighting, Warriors have a very real sense of danger around them.
Whether they are fast and lean or brutally strong fighters does matter little, their posture tell the same story: 
I am ready to kill you with a spoon, so do not mess with me.}
\begin{itemize}
	\item \textbf{Aptitudes:} Power, Offense
	\item \textbf{Common Attributes:} Power, Instinct, Finesse 
	\item \textbf{\nameref{Tags}:} PA
	\item \textbf{Specialties:} Cleave, Combat Trick, Fencing, Hammerblow, Lightning Strike, Piercing Strike
	\item \textbf{Synergies:} Athlete, Observer
	\item \textbf{Sample Consequences:} 
	\begin{itemize}
		\item Glancing Hit: While the attack does connect with the target, it is a close call and the character just cannot bring his full power to bear. Target suffers half damage.
		\item Overbalance: The character timed his swing in a bad way and now struggles to keep his balance, making himself an easier target until he can recover. +5 to attack the Character for one turn.
	\end{itemize}
	\item \textbf{Common Modifiers:}
	\begin{itemize}
		\item Difficult Terrain: -5 (or worse)
		\item Target Size
	\end{itemize}
\end{itemize}

% section profession (end)

\section{Specialties}% (fold)
\label{sec:Specialties}
Specialties represent a tightly defined action and are the central resolution mechanic used in Grimdark. 
Every Specialty is associated with a Profession.
Thus, Specialties are usually written like this: Pretty Words (Spokesman).
That is the name of a skill is followed by the respective Profession in parenthesis.

\subsection{Common Usage Tags}\label{Tags}
Specialty descriptions make use of so called Common Usage Tags - or simply Tags. 
These are marker words, that match a Specialty to a certain kind of common usage.
As mentioned before, in Grimdark a single goal can be achieved by different Specialties scattered over different Professions.
Each may have its own preconditions, limitations, situational modifier or special rules, but in the end, Specialties with the same Tag can resolve a common situation - one way or another.
\begin{itemize}
	\item \textbf{Physical Attack (PA):} A physical attack is used to inflect physical harm or stress on a target. The target of a Physical Attack typically gets the chance to use a Specialty with the Physical Defense (PD) Tag (see below) to prevent or reduce the inflicted harm.
	\begin{itemize}
		\item \textit{Examples: Brawl(Athlete), Throw(Athlete), all Gunman, all Warrior}
	\end{itemize}
	\item \textbf{Physical Attack (PD):} Any kind of physical defense is used to protect the character from harm inflicted by others using a Specialty marked with the Physical Attack Tag.
	\begin{itemize}
		\item \textit{Examples: Hard Target(Acrobat), Endure(Athlete), Parry/Block(Athlete)}
	\end{itemize}
	\item \textbf{Mental Attack (MA):} A mental attack is used to either inflict mental harm to the target or influence the target to do the characters biding. The target of a Mental Attack typically gets the chance to resist by using a Specialty marked with Mental Defense (see below).
	\begin{itemize}
		\item \textit{Examples: Seduce(Spokesman), Deceive(Betrayer), Preach(Believer), Pull Rank(Commander), Terrify(Enforcer), Bribe(Trader)}
	\end{itemize}
	\item \textbf{Mental Defense (MD):} With a Specialty of the mental defense group, a character can attempt to protect himself from being influenced or mentally harmed by others using a Specialty marked with the Mental Attack Tag.
	\begin{itemize}
		\item \textit{Examples: Deny(Believer), Selfishness(Betrayer), Rationality(Rationalist), Scrutiny(Observer)}
	\end{itemize}
	\item \textbf{Support (SUP):} Specialties marked with the Support Tag are used to aid allied characters in some way.
	\begin{itemize}
		\item \textit{Examples: Combat Sense(Command), Rally(Command), Inspire(Believer), Steel(Command), Distract(Spokesman), Feint(Betrayer), Maneuvering(Acrobat), Preach(Believer), Calculus(Rationalist)}
	\end{itemize}
	\item \textbf{Problem Solving (SOL):} Specialties marked with the Problem Solving Tag can be used to solve some kind of obstacle the character may face, which is not related to either physical or mental combat, like a locked door, an encrypted data pad or a guarded area.
	\begin{itemize}
		\item \textit{Examples: Silent Move(Ghost), Sixth Sense(Observer), Hacking(Technologist), Cipher(Rationalist), Triathlete(Athlete), Leap(Acrobat), Forgery(Betrayer), Heave(Athlete)}
	\end{itemize}
\end{itemize}

\begin{DndSidebar}{Recommendation: Character has access to Tags}
It is recommended for each character to have access (that is he either has a fitting Specialty or its Profession) to at least one Specialty from each Tag. 
This way, he will have the potential to participate in nearly any situation.\\
In addition, it is highly recommended for each character to have access to at least one Specialty of each Physical and Mental Defense category.
Otherwise, players might find their characters very helpless in the face of physical or mental combat, which may end with regular death, insanity, corruption or manipulation by opponents - which may get frustrating quickly.\\
GMs should make sure to review new characters before play and talk to players should their character violates these recommendations and make sure they are aware of the issue and are ready to face the consequences or have a (good) plan to tackle such situations.
\end{DndSidebar}

\subsection{Specialty Level}
Specialties can have one of five levels. The level of a specialty directly determines the Modifier of the Specialty check. With each level, this modifier increases by 5 to a maximum of +20. Should a character have a Profession rating that is higher than is specialist modifier, he may instead use his Profession rating.
\begin{itemize}
	\item \textbf{Untrained:} This is the base of all Specialties. The character is completely untrained in this specific task and thus takes a Penalty of -10.
	\item \textbf{Trained:} The character has some training in this Specialty and thus gets a +5 bonus on the task.
	\item \textbf{Experienced:} The character is competent in this Specialty and has a good chance to succeed at most tasks. He gets a +10 bonus.
	\item \textbf{Veteran:} The character has faced this task countless times and only few others are their better. He gets a +15 bonus.
	\item \textbf{Master:} The character has mastered this Specialty and there are virtually none more competent at it. He gets a +20 bonus.
\end{itemize}

\subsection{Subtle and Blatant Specialties}
\label{skill_types}
Grimdark uses two different type of Specialties: Subtle and Blatant.
\begin{itemize}
	\item \textbf{Subtle:} These Specialties are (somewhat) covert or low-profile in manner and draw considerable less attention than Blatant Specialties. 
	Subtle Specialties produce less severe effects on a failure or partial success but in turn are slightly less likely to succeed and loose a Draw during an opposed test.\\
	Examples: Pretty Words (Spokesman), Disguise (Betrayer), Balancing (Acrobat), Snap-Shot (Gunner), Fencing (Warrior)

	\item \textbf{Blatant:} These Specialties are flashy, loud, extreme or all of them (and more) at once. 
	Using a blatant Specialty is nearly guaranteed to draw (potentially unwanted) attention.
	A Blatant Specialty will cause considerable effects on failure or partial success but has a higher chance to succeed and always win a Draw during an opposed test.\\
	Examples: Quarrel (Spokesman), Rally (Commander), Break (Athlete), Walking Fire (Gunner), Hammerblow (Warrior)
\end{itemize}
Some Specialties may be used in both Subtle or Blatant way, e.g. the Demolition (Tinkerer) skill. 
Discretely placing a bomb can be done in a subtle way, but using a charge to blow open a door and shake those within certainly can only be considered blatant.
The GM is the final arbiter when it comes to deciding whether a skill is used in a Blatant or Subtle way.


% section Specialties (end)
