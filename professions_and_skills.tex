%!TEX root = ./Grimdark.tex
\chapter{Professions \& Skills}
Grimdark's skill system is divided into so-called Professions, which group associated skills together. Raising a certain skill will also raise the profession bonus, which can be used in place of any profession skill. 
This represents the fact that specialists are also (somewhat) proficient in the general field.\\
A renown surgeon for cybernetic replacements is certainly also quite capable at first aid treatments after all, and a famous climber is quite likely a dangerous wrestler as well...

\section{Professions} % (fold)
\label{sec:professions}
A Profession represent the generalist ability of a character in a certain broad area of expertise.
Each Profession covers a range of skills which are somewhat related.
In addition, they provide a fast and loose overview of the character.
Is he a brutal underhive ganger with high scores in the Betrayer, Athlete and Fighter professions?
Or an ancient tech priest with a more accumulated knowledge in his left (augmented) toe, than a handful of highborn nobles put together, reflected by his  Tinkerer, Technologist and Logic professions?
\begin{figure}[ht]
	\begin{DndSidebar}{Profession Example}
	\textit{Sister Aryanna of the Ordo Dialogis and her death-cult comrade Krill'Ta are attempting to infiltrate a gathering of noblewomen of the Hive World Olrabus, suspecting the machinations of a genestealer cult in its midst. As a distrustful guard is denying them entrance, a hasty debate between the two members of the Inquisitors retinue using Ordo hand signs erupts. Should Aryanna attempt to tell a convoluted (and utterly crafted) story to convince the guard - or should Krill'Ta attempt to seduce the guard with her death world charm?\\
	Both would be valid approaches, but seducing the guard may take Krill'Ta out of the next scene and leave the sister alone in the midst of the noblewomen. On the other hand, sister Aryanna isn't particular known for her abilities to lie - an ability which is far from well-perceived by her Ordo after all...\\}
	\end{DndSidebar}
\end{figure}

One central decision of Grimdark, contrary to many other game systems, is the fact that it does not assume that player characters or even a group of player characters, cover all or even most professions. 
There are many different ways to tackle any given situation scattered over the different professions that may appear redundant on a purely mechanical take.
But each profession has its very own take and spin on "their" solution to a problem, along with certain role-play implications, limitations and situational modifiers.
Thus a certain given situation may be much easier to solve by one profession than by another - yet they all have their uses, without forcing someone to break out of character to tackle a certain solution.

On a mechanical level, Professions work (mostly) like skills.
You can use them instead of any related skill.
The main difference between skills and Professions, is the fact that you cannot raise Professions directly.
The roll modifier is instead determined by the current Profession rating (see below).

\begin{figure}[hb]
	\begin{DndReadAloud}
	\textbf{Design Decision: Professions}
	\begin{itemize}
		\item \textbf{Facilitate Player Creativity}: If a player has an idea, which is not covered by the existing skills, it is usually easy to discern the Profession to which the described action relates. Having the option to roll a Profession check in such a case, allows for great player creativity. GM and Players are encouraged to develop new skills for commonly used actions of players that are not foreseen by Grimdark.
		\item \textbf{Make Everyone a Generalist}: In many skill-based game systems, players may end up with specialist characters which are very proficient in skill A but absolutely untrained in skill B, although the two skills are actually quite closely related. This does not make a whole lot of sense, as every specialist is also necessarily a generalist in his field of purview - at least to some degree.
		\end{itemize}
	\end{DndReadAloud}
\end{figure}

\subsection{The Profession Rating}
Each profession has a certain Profession rating, representing the total ability of the character in that Profession. 
This rating \textbf{increases by 2 for each new skill level learned} in an associated skill of the profession.

\subsection{The Profession Check}
A Profession check works just like a normal skill check with a single difference.
Instead of using the respective skill modifier, it uses the Profession Rating.
Everything else works the same, including the used attributes and situational modifiers of the respective skill.

\begin{figure}[hb]
	\begin{DndSidebar}{Profession Check Example}
	\textit{Inquisitor Otto Brecht has been trapped into a hive corridor with all doors locked. The straight forward Inquisitors decides to try and force open one of the doors with his own strength, hoping to still remain his covert approach that way. If that fails, he can still blast open the doors with a grenade after all ...\\
	Brecht is not particularly skilled at breaking things- he relies on the brute Nic'Or for such things - but he is keeping himself in top physical condition and is tough as nail. He should have a good enough chance to break open the doors...}\\
	That is, Brecht does not have the Break skill of the Athlete Profession, but he does have two ranks each in the Triathlete and Endure skills of that Profession. Thus, his Athlete Profession rating is: $2*4=8$.
		\begin{itemize}
		\item \textbf{Skill / Profession:} Break / Athlete
		\item \textbf{Skill Modifier:} 0 (untrained => use Profession)
		\item \textbf{Profession Rating:} 8
		\item \textbf{Attributes:} Strength 37, Toughness 44
		\item \textbf{Modifiers:} Locked Door (-10)
		\item \textbf{Target Numbers:} 35 (37+8-10), 42 (44+8-10)
		\item \textbf{Rolls:} 77 (no pass), 43 (no pass)
		\item \textbf{Result:} Failure
	\end{itemize}
	\textit{But the door won't budge. Going loud it is then. So be it. He never liked sneaking around anyway...}
	\end{DndSidebar}
\end{figure}

\subsection{The Professions}
Grimdark uses the following professions:
\begin{itemize}
	\item Acrobat
	\item Athlete
	\item Believer
	\item Betrayer
	\item Commander
	\item Enforcer
	\item Fighter
	\item Ghost
	\item Gunner
	\item Medic
	\item Navigator
	\item Observer
	\item Operator
	\item Psyker
	\item Rationalist
	\item Spokesman
	\item Survivalist
	\item Technologist
	\item Tinkerer
	\item Trader
\end{itemize}

% section profession (end)

\section{Skills}% (fold)
\label{sec:skills}
Skills represent narrow, specialist activities 

% section skills (end)