%!TEX root = ./Grimdark.tex
\section{Specialties}% (fold)
\label{sec:Specialties}
Specialties represent a tightly defined action and are the central resolution mechanic used in Grimdark. 
Every Specialty is associated with a Profession.
Thus, Specialties are usually written like this: Pretty Words (Spokesman).
That is the name of a skill is followed by the respective Profession in parenthesis.

\subsection{Common Usage Tags}\label{Tags}
Specialty descriptions make use of so called Common Usage Tags - or simply Tags. 
These are keywords, that match a Specialty to a certain kind of common usage.
As mentioned before, in Grimdark a single goal can be achieved by different Specialties scattered over different Professions.
Each may have its own preconditions, limitations, situational modifier or special rules, but in the end, Specialties with the same Tag can resolve a common situation - one way or another.
\begin{itemize}
	\item \textbf{Physical Attack (PA):} A physical attack is used to inflict physical harm or stress on a target. The target of a Physical Attack typically gets the chance to use a Specialty with the Physical Defense Tag (see below) to prevent or reduce the inflicted harm.
	\begin{itemize}
		\item \textit{Examples: Brawl(Athlete), Throw(Athlete), all Gunman, all Warrior}
	\end{itemize}
	\item \textbf{Physical Attack (PD):} Any kind of physical defense is used to protect the character from harm inflicted by others using a Specialty marked with the Physical Attack Tag.
	\begin{itemize}
		\item \textit{Examples: Hard Target(Acrobat), Endure(Athlete), Parry/Block(Athlete)}
	\end{itemize}
	\item \textbf{Mental Attack (MA):} A mental attack is used to either inflict mental harm to the target or influence the target to do the characters biding. The target of a Mental Attack typically gets the chance to resist by using a Specialty marked with Mental Defense (see below).
	\begin{itemize}
		\item \textit{Examples: Seduce(Spokesman), Deceive(Betrayer), Preach(Believer), Pull Rank(Commander), Terrify(Enforcer), Bribe(Trader)}
	\end{itemize}
	\item \textbf{Mental Defense (MD):} With a Specialty of the mental defense group, a character can attempt to protect himself from being influenced or mentally harmed by others using a Specialty marked with the Mental Attack Tag.
	\begin{itemize}
		\item \textit{Examples: Deny(Believer), Selfishness(Betrayer), Rationality(Rationalist), Scrutiny(Observer)}
	\end{itemize}
		\item \textbf{Recovery (REC):} Specialties marked with the Recovery Tag can be used to recover from stress and some may be able to treat wounds. Some only work for the character while others work for allies as well.
	\begin{itemize}
		\item \textit{Examples: First Aid(Medic), Endure(Athlete), Rally(Commander), Meditate(Believer)}
	\end{itemize}
	\item \textbf{Support (SUP):} Specialties marked with the Support Tag are used to aid allied characters in some way.
	\begin{itemize}
		\item \textit{Examples: Combat Sense(Command), Rally(Command), Inspire(Believer), Steel(Command), Distract(Spokesman), Feint(Betrayer), Maneuvering(Acrobat), Preach(Believer), Calculus(Rationalist)}
	\end{itemize}
	\item \textbf{Problem Solving (SOL):} Specialties marked with the Problem Solving Tag can be used to solve some kind of obstacle the character may face, which is not related to either physical or mental combat, like a locked door, an encrypted data pad or a guarded area.
	\begin{itemize}
		\item \textit{Examples: Sneak(Ghost), Sixth Sense(Observer), Hacking(Technologist), Cipher(Rationalist), Triathlete(Athlete), Leap(Acrobat), Forgery(Betrayer), Heave(Athlete)}
	\end{itemize}
\end{itemize}

\begin{DndSidebar}{Recommendation: Character has access to Tags}
It is recommended for each character to have access (that is he either has a fitting Specialty or its Profession) to at least one Specialty from each Tag. 
This way, he will have the potential to participate in nearly any situation.\\\noindent
In addition, it is highly recommended for each character to have access to at least one Specialty of each Physical and Mental Defense category.
Otherwise, players might find their characters very helpless in the face of physical or mental combat, which may end with regular death, insanity, corruption or manipulation by opponents - which may get frustrating quickly.\\
\noindent
GMs should make sure to review new characters before play and talk to players should their character violates these recommendations and make sure they are aware of the issue and are ready to face the consequences or have a (good) plan to tackle such situations.
\end{DndSidebar}

\subsection{Specialty Level}\label{skill_level}
Specialties can have one of five levels. The level of a specialty directly determines the Modifier of the Specialty check. With each level, this modifier increases by 5 to a maximum of +15. Should a character have a Profession rating that is higher than his specialist modifier, he may instead use his Profession rating.
\begin{itemize}
	\item \textbf{Untrained:} This is the base of all Specialties. The character is completely untrained in this specific task and thus takes a Penalty of -10.
	\item \textbf{Trained:} The character has some training in this Specialty and thus gets a +5 bonus on the task.
	\item \textbf{Experienced:} The character is competent in this Specialty and has a good chance to succeed at most tasks. He gets a +10 bonus.
	\item \textbf{Veteran:} The character has faced this task countless times and only few others are their better. He gets a +15 bonus.
	\item \textbf{Master:} The character has mastered this Specialty and there are virtually none more competent at it. He gets a +15 bonus and may re-roll once.
\end{itemize}

\subsection{Subtle and Blatant Specialties}
\label{skill_types}
Grimdark uses two different type of Specialties: Subtle and Blatant.
\begin{itemize}
	\item \textbf{Subtle:} These Specialties are (somewhat) covert or low-profile in manner and draw considerable less attention than Blatant Specialties. 
	Subtle Specialties produce less severe effects on a failure or partial success but in turn are slightly less likely to succeed and loose a Draw during an opposed test.\\
	Examples: Pretty Words (Spokesman), Disguise (Betrayer), Balancing (Acrobat), Snap-Shot (Gunner), Fencing (Warrior)

	\item \textbf{Blatant:} These Specialties are flashy, loud, extreme or all of them (and more) at once. 
	Using a blatant Specialty is nearly guaranteed to draw (potentially unwanted) attention.
	A Blatant Specialty will cause considerable effects on failure or partial success but has a higher chance to succeed and always win a Draw during an opposed test.\\
	Examples: Quarrel (Spokesman), Rally (Commander), Break (Athlete), Walking Fire (Gunner), Hammerblow (Warrior)
\end{itemize}
Some Specialties may be used in both Subtle or Blatant way, e.g. the Demolition (Tinkerer) skill. 
Discretely placing a bomb can be done in a subtle way, but using a charge to blow open a door and shake those within certainly can only be considered blatant.
The GM is the final arbiter when it comes to deciding whether a skill is used in a Blatant or Subtle way.

\onecolumn
%!TEX root = ./Grimdark.tex
\subsection{Specialties Overview Table}
\begin{DndTable}[width=\textwidth]{XXXXXXX}
\textbf{Specialty} & \textbf{Profession} & \textbf{Roll} & \textbf{Blatant} & \textbf{Tags} & \textbf{Aptitude 1} & \textbf{Aptitude 2} \\
Advanced Crafting & Technologist & Finesse+Smarts & SOL & B & Smarts & Tech \\
Aircraft & Operator & Finesse+Instinct & SOL, SUP & vary & Finesse & Offense \\
Balancing & Acrobat & Finesse+Instinct & PD, SOL & S & Finesse & Defense \\
Battle Plan & Commander & Instinct+Smarts & SOL, SUP & S & Smarts & Leadership \\
Blackmail & Enforcer & Power+Charisma & MA & vary & Power & Social \\
Block & Athlete & Power+Toughness & PD & B & Power & Defense \\
Bookkeeping & Trader & Smarts+Instinct & SOL & S & Smarts & Knowledge \\
Brawl & Athlete & Power+Finesse & PA & vary & Power & Offense \\
Break & Athlete & Power+Toughness & SOL & B & Power & Offense \\
Bribe & Trader & Charisma+Smarts & MA & S & Charisma, Smarts & Subtlety \\
Calculus & Rationalist & Smarts+Will & SOL, SUP & S & Smarts & Knowledge \\
Central Mass & Gunman & Finesse+Instinct & PA & B & Finesse & Offense \\
Cipher & Rationalist & Smarts+Instinct & SOL & S & Smarts & Knowledge \\
Classic Education & Loremaster & Smarts+Will & SOL & S & Smarts & Knowledge \\
Cleave & Warrior & Power+Toughness & PA & B & Power & Offense \\
Combat Sense & Commander & Instinct+Smarts & PD, SUP & S & Instinct & Defense \\
Combat Trick & Warrior & Power+Finesse & PA, SUP & S & Finesse, Power & Subtlety \\
Common Crafting & Tinkerer & Finesse+Smarts & SOL & B & Finesse & Fieldcraft \\
Communion & Technologist & Will+Smarts & SOL & S & Will & Tech \\
Contortionist & Acrobat & Finesse+Toughness & SOL & S & Finesse & Subtlety \\
Cooking & Survivalist & Finesse+Instinct & SUP & B & Instinct & Fielcraft \\
Dancing & Acrobat & Finesse+Toughness & SOL & S & Finesse & Social \\
Deceive & Betrayer & Charisma+Instinct & MA & S & Charisma & Social \\
Demolition & Tinkerer & Finesse+Will & PA, SOL & vary & Finesse & Fieldcraft \\
Deny & Believer & Will+Toughness & MD & B & Will & Defense \\
Diagnose & Medic & Instinct+Smarts & SOL, SUP & S & Instinct & Knowledge \\
Disguise & Betrayer & Finesse+Charisma & SOL & S & Finesse 	Social \\
Distract & Spokesman & Charisma+Instinct & MA, SOL, SUP & S & Charisma & Subtlety, Social \\
Enclosed & Navigator & Instinct+Smarts & SOL & S & Instinct & Fieldcraft \\
Endure & Athlete & Toughness+Will & PD, REC & B & Toughness & Defense \\
Estimate & Trader & Instinct+Smarts & SOL & S & Smarts, Instinct & Knowledge \\
Feint & Betrayer & Finesse+Will & MA, SUP & S & Finesse & Subtlety \\
Fencing & Warrior & Finesse+Toughness & PA & S & Finesse & Subtlety \\
First Aid & Medic & Finesse+Smarts & REC, SUP & B & Finesse & Fieldcraft \\
Foraging & Survivalist & Instinct+Smarts & SOL & B & Instinct & Fieldcraft \\
Forgery & Betrayer & Finesse+Smarts & SOL & S & Finesse & Fieldcraft \\
Ghost Move & Ghost & Finesse+Instinct & SOL & S & Finesse & Subtlety  \\
Hacking & Technologist & Smarts+Will & SOL & S & Smarts & Tech \\
Haggle & Trader & Charisma+Power & SOL & B & Charisma & Social \\
Hammerblow & Warrior & Power+Toughness & PA & B & Power & Offense  \\
Hard Target & Acrobat & Instinct+Finesse & PD & B & Instinct & Defense \\
Hide & Ghost & Finesse+Will & SOL & S & Finesse, Will & Subtlety  \\
Imperium of Man & Loremaster & Smarts+Will & SOL & S & Smarts & Knowledge  \\
Inquiry & Spokesman & Charisma+Instinct & SOL & vary & Charisma & Social \\
Inspire & Believer & Charisma+Smarts & SUP, SOL & vary & Charisma & Leadership \\
Interrogate & Enforcer & Power+Charisma & MA, SOL & vary & Power, Charisma & Offense \\
Leap & Acrobat & Finesse+Power & PD, SOL & vary & Finesse, Power & Defense \\
Lightning Strike & Warrior & Power+Finesse & PA & B & Power & Offense \\
Linguistic & Rationalist & Smarts+Instinct & SOL & vary & Smarts & Knowledge \\
Lip Reading & Observer & Instinct+Smarts & SOL & S & Instinct & Subtlety  \\
Lockpicking & Tinkerer & Finesse+Instinct & SOL & S & Finesse & Fieldcraft \\
Maneuvering & Acrobat & Finesse+Instinct & SUP & S & Finesse & Subtlety  \\
Medical Crafting & Medic & Smarts+Finesse & SUP, SOL & B & Smarts & Fieldcraft, Knowledge \\
Meditate & Believer & Will+Toughness & REC & S & Will & Defense \\
\end{DndTable}

\begin{DndTable}[width=\textwidth]{XXXXXXX}
\textbf{Specialty} & \textbf{Profession} & \textbf{Roll} & \textbf{Blatant} & \textbf{Tags} & \textbf{Aptitude 1} & \textbf{Aptitude 2} \\
Occult & Loremaster & Will+Smarts & SOL & S & Will, Smarts & Knowledge \\
Personal & Operator & Finesse+Instinct & SOL & vary & Finesse, Instinct & Subtlety  \\
Piercing Strike & Warrior & Finesse+Instinct & PA & S & Finesse & Subtlety \\
Preach & Believer & Charisma+Will & MA, REC, SUP, SOL & B & Charisma, Will & Social \\
Pretty Words & Spokesman & Charisma+Smarts & MA & S & Charisma & Social \\
Primitive Crafting & Survivalist & Instinct+Finesse & SOL, SUP & B & Instinct & Fieldcraft \\
Pull Rank & Commander & Charisma+Power & MA, SOL & B & Charisma & Leadership \\
Quarrel & Spokesman & Charisma+Smarts & MA, SOL & B & Charisma, Smarts & Social \\
Rally & Commander & Charisma+Will & MD, REC, SUP & B & Charisma & Leadership, Defense \\
Rationality & Rationalist & Smarts+Will & MD, REC, SUP, SOL & S & Smarts & Defense \\
Rehabilitate & Medic & Charisma+Smarts & REC, SUP & B & Charisma & Social \\
Rites of Activation & Technologist & Smarts+Will & SOL & B & Smarts & Tech \\
Rites of Maintenance & Technologist & Smarts+Finesse & REC, SOL & B & Smarts, Finesse & Tech \\
Sanctify & Believer & Will+Power & MA, SUP, SOL & B & Will & Fieldcraft \\
Scrutiny & Observer & Instinct+Will & MD, SOL & S & Instinct & Social \\
Search & Observer & Instinct+Will & SOL & B & Instinct & Fieldcraft \\
Seduce & Spokesman & Charisma+Power & SOL & vary & Charisma & Social \\
Selfishness & Betrayer & Will+Toughness & MD & S & Will & Defense \\
Shadow Organizations & Loremaster & Smarts+Instinct & SOL & S & Smarts & Knowledge \\
Shadowing & Ghost & Finesse+Instinct & SOL & S & Finesse, Instinct & Subtlety \\
Sixth Sense & Observer & Instinct+Finesse & PD, SOL & S & Instinct & Defense \\
Sleight of Hand & Betrayer & Finesse+Will & SOL & S & Finesse & Subtlety \\
Smuggle & Betrayer & Smarts+Finesse & SOL & S & Smarts & Subtlety \\
Snap Shot & Gunman & Finesse+Instinct & PA & S & Finesse & Offense \\
Sniping & Gunman & Finesse+Will & PA & S & Finesse & Subtlety \\
Steel & Commander & Charisma+Power & MD, SUP & vary & Charisma & Leadership \\
Surface & Navigator & Instinct+Smarts & SUP, SOL & S & Instinct & Knowledge \\
Surgery & Medic & Finesse+Smarts & REC, SUP, SOL & B & Finesse & Knowledge \\
Sweeping Fire & Gunman & Power+Instinct & PA & B & Power & Offense \\
Tactica Immperialis & Loremaster & Smarts+Instinct & SUP, SOL & S & Smarts & Knowledge \\
Taming & Survivalist & Instinct+Charisma & SUP, SOL & vary & Instinct, Power & Leadership \\
Technomat & Tinkerer & Instinct+Smarts & REC, SUP, SOL & vary & Instinct, Smarts & Fieldcraft \\
Terrify & Enforcer & Power+Charisma & MA, SUP & B & Power & Offense \\
Throw & Athlete & Power+Finesse & PA, SOL & vary & Power & Offense \\
Torture & Enforcer & Finesse+Power & MA, SOL & B & Finesse, Power & Offense \\
Tracked & Operator & Finesse+Instinct & SOL & vary & Finesse & Offense \\
Triathlete & Athlete & Toughness+Power & SOL & vary & Toughness, Power & Offense \\
Trick Shot & Gunman & Finesse+Instinct & PA, SOL & S & Finesse & Subtlety \\
Vanish & Ghost & Finesse+Smarts & SOL & S & Finesse & Subtlety \\
Void & Operator & Finesse+Smarts & SOL & B & Finesse, Smarts & Offense \\
Void Nav & Navigator & Smarts+Instinct & SOL, SUP & B & Smarts & Knowledge \\
Walker & Operator & Finesse+Will & SOL & B & Finesse & Offense \\
Walking Fire & Gunman & Power+Finesse & PA & B & Power & Offense \\
Warcry & Enforcer & Power+Will & MA, SUP & B & Power & Offense, Leadership \\
Warp & Navigator & Will+Smarrs & SUP, SOL & S & Will & Knowledge, Psyker \\
Ways of Mars & Loremaster & Smarts+Will & SUP, SOL & S & Smarts & Knowledge, Tech \\
Wheeled & Operator & Finesse+Instinct & SUP, SOL & B & Finesse & Offense \\
Xeno & Operator & Instinct+Smarts & SUP, SOL & B & Instinct & Offense \\
Xenology & Loremaster & Smarts+Instinct & SUP, SOL & S & Smarts & Knowledge \\
Zoology & Survivalist & Smarts+Instinct & REC, SOL & S & Smarts & Knowledge \\
\end{DndTable}
\twocolumn
% section Specialties (end)
